\documentclass[12pt]{report}

\usepackage{cmap}
\usepackage[T1,T2A]{fontenc}
\usepackage[utf8]{inputenc}
\usepackage[english, russian]{babel}
\usepackage{amssymb}
\usepackage{amsmath}
\usepackage{amsthm}
\usepackage{dsfont}
\usepackage{bm}
\usepackage{diagbox}
\usepackage{array}
\usepackage{placeins}
\usepackage[left=20mm,right=10mm,top=20mm,bottom=20mm,bindingoffset=2mm]{geometry}
\usepackage{indentfirst}
\usepackage[utf8]{inputenc}
\usepackage{float}
\usepackage[hidelinks]{hyperref}
\usepackage{graphicx}
\usepackage{xcolor}
\usepackage{listings}
\usepackage{enumitem}
\usepackage{longtable}
\usepackage{booktabs}
\usepackage{calc}

\DeclareMathOperator{\N}{\mathbb{N}}
\DeclareMathOperator{\R}{\mathbb{R}}
\DeclareMathOperator{\Z}{\mathbb{Z}}
\DeclareMathOperator{\CC}{\mathbb{C}}
\DeclareMathOperator{\PP}{\mathrm{P}}
\DeclareMathOperator{\Expec}{\mathrm{E}}
\DeclareMathOperator{\Var}{\mathrm{Var}}
\DeclareMathOperator{\Cov}{\mathrm{Cov}}
\DeclareMathOperator{\asConv}{\xrightarrow{a.s.}}
\DeclareMathOperator{\LpConv}{\xrightarrow{Lp}}
\DeclareMathOperator{\pConv}{\xrightarrow{p}}
\DeclareMathOperator{\dConv}{\xrightarrow{d}}

\hypersetup{
	colorlinks=true,
	linkcolor=blue,
	citecolor=blue,
	urlcolor=blue
}

\lstset{language=Python, extendedchars=\true}

\lstdefinestyle{pythonstyle}{
	language=Python,
	backgroundcolor=\color{lightgray},
	commentstyle=\color{green},
	keywordstyle=\color{blue},
	stringstyle=\color{red},
	basicstyle=\ttfamily,
	frame=single,
	breaklines=true
}

\addto\captionsrussian{\renewcommand{\refname}{Список использованных источников}}

% Переопределение нумерации разделов (1, 2, 3 вместо 0.1, 0.2)
\renewcommand{\thesection}{\arabic{section}}
\renewcommand{\thesubsection}{\thesection.\arabic{subsection}}
\setcounter{section}{0}

% Переопределение команды \section для автоматического разрыва страницы
\let\oldsection\section
\renewcommand{\section}[1]{\newpage\oldsection{#1}}

\begin{document}
	
	\begin{titlepage}
		\begin{center}
			\large{Федеральное государственное автономное образовательное учреждение высшего образования <<Национальный исследовательский университет ИТМО>>}
		\end{center}
		
		\vspace{15em}
		
		\begin{center}
			\huge{\textbf{Проектная работа}} \\
			\large{По дисциплине <<Системный анализ и системное проектирование>>} \\
			\large{Текстовый отчёт по проекту}
		\end{center}
		
		\vspace{2em}
		
		\begin{flushright}
			\textit{\large{Выполнили:}} \\
			\large{Студент группы P3306} \\
			\large{Михайлов Дмитрий} \\
			\large{Андреевич} \\
			
			\large{Студент группы P3312} \\
			\large{Малышев Никита} \\
			\large{Александрович} \\
			\textit{\large{Преподаватель:}} \\
			\large{Маркина Татьяна} \\
			\large{Анатольевна}
		\end{flushright}
		
		\vspace{2cm}
		
		\begin{figure}[h]
			\centering
			\includegraphics[width=0.5\linewidth]{Lab1_sa_sd/image.png}
		\end{figure}
		
		\begin{center}
			Санкт-Петербург \\
			2025 год
		\end{center}
	\end{titlepage}
	
	\tableofcontents
	\newpage
	
	\section{Введение}
	
	Данный проект посвящён разработке информационной системы для управления библиотекой. Система предназначена для автоматизации процессов выдачи и возврата книг, управления каталогом библиотечных материалов, бронирования и продления сроков выдачи, а также для предоставления читателям удобного доступа к библиотечным ресурсам.
	
	Система решает проблему неэффективного управления библиотечными процессами, связанную с ручной обработкой операций выдачи и возврата книг, отсутствием централизованного каталога и сложностью отслеживания состояния библиотечных материалов. Автоматизация этих процессов позволит повысить эффективность работы библиотеки, улучшить качество обслуживания читателей и обеспечить более точный учёт библиотечного фонда.
	
	Основными пользователями системы являются:
	\begin{itemize}
		\item \textbf{Читатели} --- пользователи библиотеки, которые могут искать книги в каталоге, бронировать и продлевать сроки выдачи, просматривать свою историю выдачи;
		\item \textbf{Библиотекари} --- сотрудники библиотеки, которые управляют каталогом, обрабатывают операции выдачи и возврата, управляют пользователями;
		\item \textbf{Администраторы} --- управляют системой, настраивают параметры, формируют отчёты.
	\end{itemize}
	
	\section{Бизнес-требования}
	
	Бизнес-требования к системе сформулированы в формате SMART (Specific, Measurable, Achievable, Relevant, Time-bound):
	
	\begin{itemize}
		\item Увеличить среднее количество бронирований на пользователя на \textbf{15\%} через персональные рекомендации и напоминания.
		
		\item Время ответа поискового запроса не более \texttt{2 секунд} для \textbf{95\%} запросов.
		
		\item Доля успешных резерваций не менее \textbf{99\%} от впервые отправленных запросов.
		
		\item Предоставлять ежемесячные дашборды по выдаче, продлениям, популярности жанров и периоду пиковых нагрузок.
		
		\item Внедрить автоматизированные уведомления за \textbf{3} дня до просрочки, чтобы заполучить лояльность пользователей.
	\end{itemize}
	
	\section{Анализ и проектирование}
	
	\subsection{Итерация 1: Карта стейкхолдеров и план выявления требований}
	
	\subsubsection{Карта стейкхолдеров}
	
	На первом этапе проекта была проведена идентификация и анализ стейкхолдеров системы. Карта стейкхолдеров представлена в таблице~\ref{tab:stakeholders}.
	
	\begin{table}[h]
		\hspace{-1.5cm}
		\begin{tabular}{|l|l|c|c|c|}
			\hline
			\textbf{Стейкхолдер} & \textbf{Роль} & \textbf{Внутренний/внешний} & \textbf{Влияние} & \textbf{Интерес} \\
			\hline
			Директор библиотеки & Заказчик, руководитель & Внутренний & Высокое & Высокий \\
			\hline
			Читатель (пользователь) & Пользователь системы & Внешний & Высокое & Высокий \\
			\hline
			Библиотекарь & Администратор, оператор & Внутренний & Среднее & Высокий \\
			\hline
			IT-администратор & Технический специалист & Внутренний & Среднее & Средний \\
			\hline
			Разработчик & Технический специалист & Внутренний & Высокое & Средний  \\
			\hline
			Бухгалтер & Финансовый специалист & Внутренний & Среднее & Средний  \\
			\hline
			Городская администрация & Регулятор / инвестор & Внешний & Высокое & Средний \\
			\hline
			Провайдер платежей & Сервис-партнёр & Внешний & Среднее & Низкий \\
			\hline
		\end{tabular}
		\caption{Стейкхолдеры проекта: роль, контекст взаимодействия и интересы.}
		\label{tab:stakeholders}
	\end{table}
	
	\subsubsection{План выявления требований}
	
	Для каждого стейкхолдера был определён метод выявления требований и цель опроса. План выявления требований представлен в таблице~\ref{tab:requirements}.
	
	\begin{table}[H]
		\centering
		\begin{tabular}{|>{\raggedright\arraybackslash}p{0.3\textwidth}|>{\raggedright\arraybackslash}p{0.3\textwidth}|>{\raggedright\arraybackslash}p{0.3\textwidth}|}
			\hline
			\textbf{Стейкхолдер} & \textbf{Метод} &	 \textbf{Цель опроса} \\
			\hline
			Директор библиотеки & Интервью & Выяснить стратегические цели сервиса, приоритеты по развитию, бюджетные ограничения, требования к отчетности и KPI, регуляторные и аудиторские требования \\
			\hline
			Читатель (пользователь) & Анкеты / Наблюдение & Понять, какие функции и удобства нужны читателю \\
			\hline
			Библиотекарь & Анкеты & Выяснить операционные требования: процессы выдачи/возврата, учёт материалов, каталогизация, проверки доступа, уведомления, отчётность \\
			\hline
			IT-администратор & Интервью & Определить требования к инфраструктуре, безопасности, резервному копированию, доступности сервисов, мониторингу и обновлениям, интеграциям \\
			\hline
			Разработчик & Мозговой штурм & Понять требования к API, интеграциям, архитектуре, используемым технологиям, ограничениям по времени релиза и качеству кода \\
			\hline
			Бухгалтер & Анкеты & Понять требования к учёту, финансовым операциям, интеграции с бухгалтерскими системами, отчётности \\
			\hline
			Городская администрация & Интервью & Определить требования к соответствию регуляторным нормам, взаимодействию с госструктурами, доступности услуг, инфраструктурным требованиям и вопросам открытых данных \\
			\hline
			Провайдер платежей & Интервью / Наблюдение & Понять требования к интеграции платежной инфраструктуры \\
			\hline
		\end{tabular}
		\caption{План выявления требований.}
		\label{tab:requirements}
	\end{table}
	
	\subsection{Итерация 2: BPMN-диаграмма ключевого процесса}
	
	На втором этапе была разработана BPMN-диаграмма ключевого бизнес-процесса системы --- процесса выдачи книги читателю. Диаграмма отражает последовательность действий библиотекаря и системы при обработке запроса на выдачу книги.
	
	\begin{figure}[H]
		\centering
		\includegraphics[width=1\textwidth]{Lab2_sa_sd/Lab2_sa_sd.png}
		\caption{BPMN-диаграмма процесса выдачи книги.}
		\label{fig:bpmn-issue}
	\end{figure}
	
	Также была разработана BPMN-диаграмма процесса возврата книги, которая показывает процедуру возврата книги читателем и обработки этой операции библиотекарем.
	
	\begin{figure}[H]
		\centering
		\includegraphics[width=1\textwidth]{Lab2_sa_sd/Lab2_sa_sd_returning.png}
		\caption{BPMN-диаграмма процесса возврата книги.}
		\label{fig:bpmn-return}
	\end{figure}
	
	\subsection{Итерация 3: Use Case Diagram и детализация прецедентов}
	
	На третьем этапе был разработан Use Case Diagram, который описывает основные сценарии использования системы различными акторами. Также были детализированы ключевые прецеденты использования системы.
	
	\begin{figure}[H]
		\centering
		\includegraphics[width=0.8\textwidth]{Lab3_sa_sd/Lab3_sa_sd.png}
		\caption{Use Case Diagram системы управления библиотекой.}
		\label{fig:usecase}
	\end{figure}

	\begin{longtable}[]{@{}
		>{\raggedright\arraybackslash}p{0.192\columnwidth}
		>{\raggedright\arraybackslash}p{0.200\columnwidth}
		>{\raggedright\arraybackslash}p{0.608\columnwidth}@{}}
	  \toprule
	  \multicolumn{3}{@{}>{\raggedright\arraybackslash}p{\columnwidth}@{}}{%
	  \begin{minipage}[b]{\linewidth}\raggedright
	  Авторизация (регистрация и активация учётной записи)
	  \end{minipage}} \\
	  \midrule
	  \endhead
	  \textbf{Описание} &
	  \multicolumn{2}{>{\raggedright\arraybackslash}p{0.808\columnwidth}@{}}{%
	  Определить, какие ресурсы и действия доступны конкретному пользователю в
	  рамках системы, на основе его учётной записи, ролей и прав.} \\
	  \textbf{Предусловия} &
	  \multicolumn{2}{>{\raggedright\arraybackslash}p{0.808\columnwidth}@{}}{%
	  \begin{minipage}[t]{\linewidth}\raggedright
	  \begin{enumerate}
	  \def\labelenumi{\arabic{enumi}.}
	  \item
		Пользователь не авторизован.
	  \item
		Система доступна для авторизации и регистрации новых пользователей.
	  \end{enumerate}
	  \end{minipage}} \\
	  \textbf{Акторы (Участники)} &
	  \multicolumn{2}{>{\raggedright\arraybackslash}p{0.808\columnwidth}@{}}{%
	  Неавторизованный пользователь, Система} \\
	  \multicolumn{3}{@{}>{\raggedright\arraybackslash}p{\columnwidth}@{}}{%
	  \textbf{Основной сценарий}} \\
	  \textbf{№ шага} &
	  \multicolumn{2}{>{\raggedright\arraybackslash}p{0.808\columnwidth}@{}}{%
	  \textbf{Действие}} \\
	  1 &
	  \multicolumn{2}{>{\raggedright\arraybackslash}p{0.808\columnwidth}@{}}{%
	  Неавторизованный пользователь инициирует процесс регистрации в
	  системе.} \\
	  2 &
	  \multicolumn{2}{>{\raggedright\arraybackslash}p{0.808\columnwidth}@{}}{%
	  Система запрашивает у неавторизованного пользователя регистрационные
	  данные (ФИО, электронный адрес, пароль, подтверждение пароля).} \\
	  3 &
	  \multicolumn{2}{>{\raggedright\arraybackslash}p{0.808\columnwidth}@{}}{%
	  Неавторизованный пользователь вводит запрошенные регистрационные
	  данные.} \\
	  4 &
	  \multicolumn{2}{>{\raggedright\arraybackslash}p{0.808\columnwidth}@{}}{%
	  Неавторизованный пользователь отправляет введённые регистрационные
	  данные на обработку.} \\
	  5 &
	  \multicolumn{2}{>{\raggedright\arraybackslash}p{0.808\columnwidth}@{}}{%
	  Система проверяет корректность введённых регистрационных данных.} \\
	  6 &
	  \multicolumn{2}{>{\raggedright\arraybackslash}p{0.808\columnwidth}@{}}{%
	  Система проверяет уникальность указанного электронного адреса
	  пользователя.} \\
	  7 &
	  \multicolumn{2}{>{\raggedright\arraybackslash}p{0.808\columnwidth}@{}}{%
	  Система создаёт учётную запись пользователя в состоянии «ожидает
	  подтверждения».} \\
	  8 &
	  \multicolumn{2}{>{\raggedright\arraybackslash}p{0.808\columnwidth}@{}}{%
	  Система формирует запрос на подтверждение регистрации и направляет его
	  пользователю (например, по электронной почте).} \\
	  9 &
	  \multicolumn{2}{>{\raggedright\arraybackslash}p{0.808\columnwidth}@{}}{%
	  Пользователь подтверждает регистрацию в системе в ответ на запрос.} \\
	  10 &
	  \multicolumn{2}{>{\raggedright\arraybackslash}p{0.808\columnwidth}@{}}{%
	  Система активирует учётную запись пользователя.} \\
	  \multicolumn{3}{@{}>{\raggedright\arraybackslash}p{\columnwidth}@{}}{%
	  \textbf{Альтернативный сценарий: Несовпадающие пароли (шаг №5)}} \\
	  5.А.1 &
	  \multicolumn{2}{>{\raggedright\arraybackslash}p{0.808\columnwidth}@{}}{%
	  Система определяет, что значения пароля и подтверждения пароля не
	  совпадают.} \\
	  5.А.2 &
	  \multicolumn{2}{>{\raggedright\arraybackslash}p{0.808\columnwidth}@{}}{%
	  Система информирует неавторизованного пользователя о несоответствии
	  паролей.} \\
	  5.А.3 &
	  \multicolumn{2}{>{\raggedright\arraybackslash}p{0.808\columnwidth}@{}}{%
	  Система запрашивает повторный ввод пароля и его подтверждения.} \\
	  5.А.4 &
	  \multicolumn{2}{>{\raggedright\arraybackslash}p{0.808\columnwidth}@{}}{%
	  Неавторизованный пользователь вводит новый пароль и его
	  подтверждение.} \\
	  5.А.5 &
	  \multicolumn{2}{>{\raggedright\arraybackslash}p{0.808\columnwidth}@{}}{%
	  Неавторизованный пользователь повторно отправляет регистрационные данные
	  на обработку.} \\
	  5.А.6 &
	  \multicolumn{2}{>{\raggedright\arraybackslash}p{0.808\columnwidth}@{}}{%
	  Система возвращается к шагу 5 основного сценария.} \\
	  \multicolumn{3}{@{}>{\raggedright\arraybackslash}p{\columnwidth}@{}}{%
	  \textbf{Альтернативный сценарий: Электронный адрес уже зарегистрирован
	  (шаг №6)}} \\
	  6.B.1 &
	  \multicolumn{2}{>{\raggedright\arraybackslash}p{0.808\columnwidth}@{}}{%
	  Система определяет, что указанная электронная почта уже зарегистрирована
	  в системе.} \\
	  6.B.2 &
	  \multicolumn{2}{>{\raggedright\arraybackslash}p{0.808\columnwidth}@{}}{%
	  Система информирует неавторизованного пользователя о том, что учётная
	  запись с таким электронным адресом уже существует.} \\
	  6.B.3 &
	  \multicolumn{2}{>{\raggedright\arraybackslash}p{0.808\columnwidth}@{}}{%
	  Система предлагает неавторизованному пользователю перейти к процедуре
	  входа в существующую учётную запись.} \\
	  6.B.4 &
	  \multicolumn{2}{>{\raggedright\arraybackslash}p{0.808\columnwidth}@{}}{%
	  Неавторизованный пользователь может инициировать процесс авторизации
	  (входа) вместо регистрации. Альтернативный сценарий завершается без
	  возврата к основному сценарию.} \\
	  \multicolumn{3}{@{}>{\raggedright\arraybackslash}p{\columnwidth}@{}}{%
	  \textbf{Альтернативный сценарий: Регистрация не подтверждена (шаг
	  №8)}} \\
	  8.C.1 &
	  \multicolumn{2}{>{\raggedright\arraybackslash}p{0.808\columnwidth}@{}}{%
	  Система удерживает учётную запись пользователя в состоянии «неактивна»
	  до истечения установленного срока.} \\
	  8.C.2 &
	  \multicolumn{2}{>{\raggedright\arraybackslash}p{0.808\columnwidth}@{}}{%
	  По истечении установленного срока система удаляет неактивную учётную
	  запись пользователя.} \\
	  8.C.3 &
	  \multicolumn{2}{>{\raggedright\arraybackslash}p{0.808\columnwidth}@{}}{%
	  Система информирует пользователя о невозможности использования
	  неактивированной учётной записи при повторной попытке регистрации или
	  авторизации.} \\
	  8.C.4 &
	  \multicolumn{2}{>{\raggedright\arraybackslash}p{0.808\columnwidth}@{}}{%
	  Альтернативный сценарий завершается без возврата к основному
	  сценарию.} \\
	  \multicolumn{3}{@{}>{\raggedright\arraybackslash}p{\columnwidth}@{}}{%
	  \textbf{Исключения (Ошибки, отсутствие данных для отображения)}} \\
	  E1 & Ошибка отправки e-mail. &
	  \begin{minipage}[t]{\linewidth}\raggedright
	  \begin{enumerate}
	  \def\labelenumi{\arabic{enumi}.}
	  \item
		Система фиксирует ошибку при отправке запроса на подтверждение
		регистрации.
	  \item
		Система уведомляет пользователя о невозможности отправки запроса на
		подтверждение регистрации.
	  \item
		Система предлагает пользователю повторить отправку запроса на
		подтверждение регистрации позже.
	  \end{enumerate}
	  \end{minipage} \\
	  E2 & Ошибка записи в базу данных. &
	  \begin{minipage}[t]{\linewidth}\raggedright
	  \begin{enumerate}
	  \def\labelenumi{\arabic{enumi}.}
	  \item
		Система фиксирует ошибку записи учётной записи в базу данных.
	  \item
		Система отклоняет регистрацию.
	  \item
		Система выводит пользователю сообщение о технической ошибке и
		предлагает повторить попытку позднее.
	  \end{enumerate}
	  \end{minipage} \\
	  \textbf{Постусловия} &
	  \multicolumn{2}{>{\raggedright\arraybackslash}p{0.808\columnwidth}@{}}{%
	  Учётная запись пользователя создана и активирована ИЛИ Регистрация не
	  завершена, и учётная запись удалена/не создана.} \\
	  \bottomrule
	  \end{longtable}
	  
	  \begin{longtable}[]{@{}
		>{\raggedright\arraybackslash}p{0.196\columnwidth}
		>{\raggedright\arraybackslash}p{0.365\columnwidth}
		>{\raggedright\arraybackslash}p{0.439\columnwidth}@{}}
	  \toprule
	  \multicolumn{3}{@{}>{\raggedright\arraybackslash}p{\columnwidth}@{}}{%
	  \begin{minipage}[b]{\linewidth}\raggedright
	  Поиск материалов
	  \end{minipage}} \\
	  \midrule
	  \endhead
	  \textbf{Описание} &
	  \multicolumn{2}{>{\raggedright\arraybackslash}p{0.804\columnwidth}@{}}{%
	  Поиск и просмотр информации о книгах и других материалах в каталоге
	  библиотеки.} \\
	  \textbf{Предусловия} &
	  \multicolumn{2}{>{\raggedright\arraybackslash}p{0.804\columnwidth}@{}}{%
	  \begin{minipage}[t]{\linewidth}\raggedright
	  \begin{enumerate}
	  \def\labelenumi{\arabic{enumi}.}
	  \item
		Пользователь имеет доступ к каталогу (гость или авторизованный).
	  \item
		Каталог доступен для выполнения поисковых операций.
	  \end{enumerate}
	  \end{minipage}} \\
	  \textbf{Акторы (Участники)} &
	  \multicolumn{2}{>{\raggedright\arraybackslash}p{0.804\columnwidth}@{}}{%
	  Пользователь, Система} \\
	  \multicolumn{3}{@{}>{\raggedright\arraybackslash}p{\columnwidth}@{}}{%
	  \textbf{Основной сценарий}} \\
	  \textbf{№ шага} &
	  \multicolumn{2}{>{\raggedright\arraybackslash}p{0.804\columnwidth}@{}}{%
	  \textbf{Действие}} \\
	  1 &
	  \multicolumn{2}{>{\raggedright\arraybackslash}p{0.804\columnwidth}@{}}{%
	  Пользователь инициирует поиск материалов в каталоге.} \\
	  2 &
	  \multicolumn{2}{>{\raggedright\arraybackslash}p{0.804\columnwidth}@{}}{%
	  Пользователь вводит ключевые характеристики искомых материалов
	  (например, название, автора, идентификатор).} \\
	  3 &
	  \multicolumn{2}{>{\raggedright\arraybackslash}p{0.804\columnwidth}@{}}{%
	  Пользователь отправляет запрос на поиск материалов.} \\
	  4 &
	  \multicolumn{2}{>{\raggedright\arraybackslash}p{0.804\columnwidth}@{}}{%
	  Система выполняет поиск материалов по указанным критериям.} \\
	  5 &
	  \multicolumn{2}{>{\raggedright\arraybackslash}p{0.804\columnwidth}@{}}{%
	  Система формирует список найденных материалов.} \\
	  6 &
	  \multicolumn{2}{>{\raggedright\arraybackslash}p{0.804\columnwidth}@{}}{%
	  Система отображает пользователю список найденных материалов.} \\
	  7 &
	  \multicolumn{2}{>{\raggedright\arraybackslash}p{0.804\columnwidth}@{}}{%
	  Пользователь задаёт дополнительные параметры отбора материалов
	  (фильтрацию, сортировку).} \\
	  8 &
	  \multicolumn{2}{>{\raggedright\arraybackslash}p{0.804\columnwidth}@{}}{%
	  Пользователь применяет выбранные параметры отбора.} \\
	  9 &
	  \multicolumn{2}{>{\raggedright\arraybackslash}p{0.804\columnwidth}@{}}{%
	  Система пересчитывает результаты с учётом указанных параметров
	  отбора.} \\
	  10 &
	  \multicolumn{2}{>{\raggedright\arraybackslash}p{0.804\columnwidth}@{}}{%
	  Система отображает обновлённый список найденных материалов.} \\
	  11 &
	  \multicolumn{2}{>{\raggedright\arraybackslash}p{0.804\columnwidth}@{}}{%
	  Пользователь запрашивает детальную информацию по выбранному
	  материалу.} \\
	  12 &
	  \multicolumn{2}{>{\raggedright\arraybackslash}p{0.804\columnwidth}@{}}{%
	  Система отображает пользователю детальную информацию о выбранном
	  материале.} \\
	  \multicolumn{3}{@{}>{\raggedright\arraybackslash}p{\columnwidth}@{}}{%
	  \textbf{Альтернативный сценарий: Поиск без указания критериев (шаг
	  №2)}} \\
	  2.А.1 &
	  \multicolumn{2}{>{\raggedright\arraybackslash}p{0.804\columnwidth}@{}}{%
	  Пользователь не вводит критерии поиска и отправляет запрос.} \\
	  2.А.2 &
	  \multicolumn{2}{>{\raggedright\arraybackslash}p{0.804\columnwidth}@{}}{%
	  Система определяет отсутствие критериев поиска.} \\
	  2.А.3 &
	  \multicolumn{2}{>{\raggedright\arraybackslash}p{0.804\columnwidth}@{}}{%
	  Система информирует пользователя об отсутствии введённых критериев.} \\
	  2.А.4 &
	  \multicolumn{2}{>{\raggedright\arraybackslash}p{0.804\columnwidth}@{}}{%
	  Система предлагает пользователю список популярных или рекомендованных
	  материалов.} \\
	  2.А.5 &
	  \multicolumn{2}{>{\raggedright\arraybackslash}p{0.804\columnwidth}@{}}{%
	  \begin{minipage}[t]{\linewidth}\raggedright
	  \begin{itemize}
	  \item
		Пользователь выбирает материал из предложенного списка ИЛИ
	  \item
		Пользователь возвращается к вводу критериев поиска (возврат к шагу 2
		основного сценария).
	  \end{itemize}
	  \end{minipage}} \\
	  \multicolumn{3}{@{}>{\raggedright\arraybackslash}p{\columnwidth}@{}}{%
	  \textbf{Альтернативный сценарий: Результаты поиска отсутствуют (шаг
	  №6)}} \\
	  6.B.1 &
	  \multicolumn{2}{>{\raggedright\arraybackslash}p{0.804\columnwidth}@{}}{%
	  Система не находит материалов, соответствующих заданным критериям.} \\
	  6.B.2 &
	  \multicolumn{2}{>{\raggedright\arraybackslash}p{0.804\columnwidth}@{}}{%
	  Система информирует пользователя об отсутствии результатов поиска.} \\
	  6.B.3 &
	  \multicolumn{2}{>{\raggedright\arraybackslash}p{0.804\columnwidth}@{}}{%
	  Система предлагает пользователю изменить или уточнить критерии
	  поиска.} \\
	  6.B.4 &
	  \multicolumn{2}{>{\raggedright\arraybackslash}p{0.804\columnwidth}@{}}{%
	  Пользователь изменяет или уточняет критерии поиска.} \\
	  6.B.5 &
	  \multicolumn{2}{>{\raggedright\arraybackslash}p{0.804\columnwidth}@{}}{%
	  Пользователь повторно отправляет запрос на поиск.} \\
	  6.B.6 &
	  \multicolumn{2}{>{\raggedright\arraybackslash}p{0.804\columnwidth}@{}}{%
	  Система возвращается к шагу 4 основного сценария.} \\
	  \multicolumn{3}{@{}>{\raggedright\arraybackslash}p{\columnwidth}@{}}{%
	  \textbf{Альтернативный сценарий: Некорректное сочетание параметров
	  отбора (шаг №9)}} \\
	  9.C.1 &
	  \multicolumn{2}{>{\raggedright\arraybackslash}p{0.804\columnwidth}@{}}{%
	  Система определяет, что заданные параметры отбора не могут быть
	  корректно применены.} \\
	  9.C.2 &
	  \multicolumn{2}{>{\raggedright\arraybackslash}p{0.804\columnwidth}@{}}{%
	  Система информирует пользователя о некорректности сочетания параметров
	  отбора.} \\
	  9.C.3 &
	  \multicolumn{2}{>{\raggedright\arraybackslash}p{0.804\columnwidth}@{}}{%
	  Система предлагает пользователю изменить параметры отбора.} \\
	  9.C.4 &
	  \multicolumn{2}{>{\raggedright\arraybackslash}p{0.804\columnwidth}@{}}{%
	  Пользователь изменяет параметры отбора.} \\
	  9.C.5 &
	  \multicolumn{2}{>{\raggedright\arraybackslash}p{0.804\columnwidth}@{}}{%
	  Пользователь повторно применяет параметры отбора.} \\
	  9.C.6 &
	  \multicolumn{2}{>{\raggedright\arraybackslash}p{0.804\columnwidth}@{}}{%
	  Система возвращается к шагу 9 основного сценария.} \\
	  \multicolumn{3}{@{}>{\raggedright\arraybackslash}p{\columnwidth}@{}}{%
	  \textbf{Исключения (Ошибки, отсутствие данных для отображения)}} \\
	  E1 & Ошибка связи с базой данных. & Поиск временно недоступен. \\
	  E2 & Превышено время ожидания. & Система предлагает повторить запрос. \\
	  \textbf{Постусловия} &
	  \multicolumn{2}{>{\raggedright\arraybackslash}p{0.804\columnwidth}@{}}{%
	  Отображены найденные результаты ИЛИ сообщение об их отсутствии.} \\
	  \bottomrule
	  \end{longtable}
	  
	  \begin{longtable}[]{@{}
		>{\raggedright\arraybackslash}p{0.216\columnwidth}
		>{\raggedright\arraybackslash}p{0.388\columnwidth}
		>{\raggedright\arraybackslash}p{0.396\columnwidth}@{}}
	  \toprule
	  \multicolumn{3}{@{}>{\raggedright\arraybackslash}p{\columnwidth}@{}}{%
	  \begin{minipage}[b]{\linewidth}\raggedright
	  Оформление брони
	  \end{minipage}} \\
	  \midrule
	  \endhead
	  \textbf{Описание} &
	  \multicolumn{2}{>{\raggedright\arraybackslash}p{0.784\columnwidth}@{}}{%
	  Пользователь резервирует материал для последующего получения в
	  библиотеке.} \\
	  \textbf{Предусловия} &
	  \multicolumn{2}{>{\raggedright\arraybackslash}p{0.784\columnwidth}@{}}{%
	  \begin{minipage}[t]{\linewidth}\raggedright
	  \begin{enumerate}
	  \def\labelenumi{\arabic{enumi}.}
	  \item
		Пользователь авторизован.
	  \item
		Материал доступен для бронирования (существуют экземпляры, подлежащие
		бронированию).
	  \end{enumerate}
	  \end{minipage}} \\
	  \textbf{Акторы (Участники)} &
	  \multicolumn{2}{>{\raggedright\arraybackslash}p{0.784\columnwidth}@{}}{%
	  Пользователь, Система, Библиотекарь} \\
	  \multicolumn{3}{@{}>{\raggedright\arraybackslash}p{\columnwidth}@{}}{%
	  \textbf{Основной сценарий}} \\
	  \textbf{№ шага} &
	  \multicolumn{2}{>{\raggedright\arraybackslash}p{0.784\columnwidth}@{}}{%
	  \textbf{Действие}} \\
	  1 &
	  \multicolumn{2}{>{\raggedright\arraybackslash}p{0.784\columnwidth}@{}}{%
	  Пользователь выбирает в каталоге материал, который требуется
	  забронировать.} \\
	  2 &
	  \multicolumn{2}{>{\raggedright\arraybackslash}p{0.784\columnwidth}@{}}{%
	  Пользователь инициирует бронирование выбранного материала.} \\
	  3 &
	  \multicolumn{2}{>{\raggedright\arraybackslash}p{0.784\columnwidth}@{}}{%
	  Система проверяет статус экземпляров выбранного материала (доступен,
	  выдан, забронирован).} \\
	  4 &
	  \multicolumn{2}{>{\raggedright\arraybackslash}p{0.784\columnwidth}@{}}{%
	  Система проверяет ограничения на бронирование для пользователя
	  (например, лимит активных броней).} \\
	  5 &
	  \multicolumn{2}{>{\raggedright\arraybackslash}p{0.784\columnwidth}@{}}{%
	  При отсутствии ограничений система создаёт запись бронирования материала
	  за пользователем.} \\
	  6 &
	  \multicolumn{2}{>{\raggedright\arraybackslash}p{0.784\columnwidth}@{}}{%
	  Система формирует и направляет библиотекарю уведомление о новой
	  брони.} \\
	  7 &
	  \multicolumn{2}{>{\raggedright\arraybackslash}p{0.784\columnwidth}@{}}{%
	  Система формирует и направляет пользователю уведомление об успешном
	  создании брони.} \\
	  \multicolumn{3}{@{}>{\raggedright\arraybackslash}p{\columnwidth}@{}}{%
	  \textbf{Альтернативный сценарий: Материал уже забронирован (шаг №3)}} \\
	  3.А.1 &
	  \multicolumn{2}{>{\raggedright\arraybackslash}p{0.784\columnwidth}@{}}{%
	  Система определяет, что экземпляры выбранного материала уже
	  забронированы другими пользователями.} \\
	  3.А.2 &
	  \multicolumn{2}{>{\raggedright\arraybackslash}p{0.784\columnwidth}@{}}{%
	  Система информирует пользователя о недоступности немедленного
	  бронирования.} \\
	  3.А.3 &
	  \multicolumn{2}{>{\raggedright\arraybackslash}p{0.784\columnwidth}@{}}{%
	  Система предлагает пользователю включение в очередь ожидания на данный
	  материал.} \\
	  3.А.4 &
	  \multicolumn{2}{>{\raggedright\arraybackslash}p{0.784\columnwidth}@{}}{%
	  Пользователь принимает решение встать в очередь ожидания.} \\
	  3.А.5 &
	  \multicolumn{2}{>{\raggedright\arraybackslash}p{0.784\columnwidth}@{}}{%
	  Система создаёт запись о включении пользователя в очередь ожидания.} \\
	  3.А.6 &
	  \multicolumn{2}{>{\raggedright\arraybackslash}p{0.784\columnwidth}@{}}{%
	  Альтернативный сценарий завершается без продолжения основного
	  сценария.} \\
	  \multicolumn{3}{@{}>{\raggedright\arraybackslash}p{\columnwidth}@{}}{%
	  \textbf{Альтернативный сценарий: Превышен лимит бронирований (шаг
	  №4)}} \\
	  4.B.1 &
	  \multicolumn{2}{>{\raggedright\arraybackslash}p{0.784\columnwidth}@{}}{%
	  Система определяет, что пользователь превысил допустимый лимит активных
	  бронирований.} \\
	  4.B.2 &
	  \multicolumn{2}{>{\raggedright\arraybackslash}p{0.784\columnwidth}@{}}{%
	  Система информирует пользователя о невозможности создания новой брони
	  из-за превышения лимита.} \\
	  4.B.3 &
	  \multicolumn{2}{>{\raggedright\arraybackslash}p{0.784\columnwidth}@{}}{%
	  Альтернативный сценарий завершается без продолжения основного
	  сценария.} \\
	  \multicolumn{3}{@{}>{\raggedright\arraybackslash}p{\columnwidth}@{}}{%
	  \textbf{Альтернативный сценарий: Отмена бронирования до подтверждения
	  библиотекарем (шаг №5)}} \\
	  5.C.1 &
	  \multicolumn{2}{>{\raggedright\arraybackslash}p{0.784\columnwidth}@{}}{%
	  Пользователь инициирует отмену ранее созданной брони.} \\
	  5.C.2 &
	  \multicolumn{2}{>{\raggedright\arraybackslash}p{0.784\columnwidth}@{}}{%
	  Система удаляет запись о бронировании материалов за пользователем.} \\
	  5.C.3 &
	  \multicolumn{2}{>{\raggedright\arraybackslash}p{0.784\columnwidth}@{}}{%
	  Система формирует и направляет библиотекарю уведомление об отмене
	  брони.} \\
	  5.C.4 &
	  \multicolumn{2}{>{\raggedright\arraybackslash}p{0.784\columnwidth}@{}}{%
	  Система формирует и направляет пользователю уведомление об успешной
	  отмене брони.} \\
	  5.C.5 &
	  \multicolumn{2}{>{\raggedright\arraybackslash}p{0.784\columnwidth}@{}}{%
	  Альтернативный сценарий завершается без возврата к основному
	  сценарию.} \\
	  \multicolumn{3}{@{}>{\raggedright\arraybackslash}p{\columnwidth}@{}}{%
	  \textbf{Исключения (Ошибки, отсутствие данных для отображения)}} \\
	  E1 & Ошибка записи в базу данных. & Бронь не создана. \\
	  E2 & Потеря соединения с сервером при бронировании. & Пользователь
	  уведомляется, и операция отменяется. \\
	  \textbf{Постусловия} &
	  \multicolumn{2}{>{\raggedright\arraybackslash}p{0.784\columnwidth}@{}}{%
	  Создана активная бронь или поставлена очередь на материал.} \\
	  \bottomrule
	  \end{longtable}
	  
	  \begin{longtable}[]{@{}
		>{\raggedright\arraybackslash}p{0.203\columnwidth}
		>{\raggedright\arraybackslash}p{0.489\columnwidth}
		>{\raggedright\arraybackslash}p{0.308\columnwidth}@{}}
	  \toprule
	  \multicolumn{3}{@{}>{\raggedright\arraybackslash}p{\columnwidth}@{}}{%
	  \begin{minipage}[b]{\linewidth}\raggedright
	  Оформление выдачи материала
	  \end{minipage}} \\
	  \midrule
	  \endhead
	  \textbf{Описание} &
	  \multicolumn{2}{>{\raggedright\arraybackslash}p{0.797\columnwidth}@{}}{%
	  Библиотекарь оформляет выдачу материала пользователю.} \\
	  \textbf{Предусловия} &
	  \multicolumn{2}{>{\raggedright\arraybackslash}p{0.797\columnwidth}@{}}{%
	  \begin{minipage}[t]{\linewidth}\raggedright
	  \begin{enumerate}
	  \def\labelenumi{\arabic{enumi}.}
	  \item
		Пользователь зарегистрирован в системе.
	  \item
		У пользователя отсутствуют блокирующие ограничения (например, активные
		штрафы, критические просрочки).
	  \item
		Материал доступен для выдачи.
	  \end{enumerate}
	  \end{minipage}} \\
	  \textbf{Акторы (Участники)} &
	  \multicolumn{2}{>{\raggedright\arraybackslash}p{0.797\columnwidth}@{}}{%
	  Пользователь, Система, Библиотекарь} \\
	  \multicolumn{3}{@{}>{\raggedright\arraybackslash}p{\columnwidth}@{}}{%
	  \textbf{Основной сценарий}} \\
	  \textbf{№ шага} &
	  \multicolumn{2}{>{\raggedright\arraybackslash}p{0.797\columnwidth}@{}}{%
	  \textbf{Действие}} \\
	  1 &
	  \multicolumn{2}{>{\raggedright\arraybackslash}p{0.797\columnwidth}@{}}{%
	  Библиотекарь идентифицирует пользователя в системе.} \\
	  2 &
	  \multicolumn{2}{>{\raggedright\arraybackslash}p{0.797\columnwidth}@{}}{%
	  Библиотекарь выбирает материал, подлежащий выдаче пользователю.} \\
	  3 &
	  \multicolumn{2}{>{\raggedright\arraybackslash}p{0.797\columnwidth}@{}}{%
	  Система проверяет доступность выбранного экземпляра материала.} \\
	  4 &
	  \multicolumn{2}{>{\raggedright\arraybackslash}p{0.797\columnwidth}@{}}{%
	  Система проверяет наличие у пользователя ограничений на выдачу (активные
	  штрафы, просроченные материалы и др.).} \\
	  5 &
	  \multicolumn{2}{>{\raggedright\arraybackslash}p{0.797\columnwidth}@{}}{%
	  При отсутствии ограничений система регистрирует выдачу материала
	  пользователю с указанием срока возврата и места выдачи.} \\
	  6 &
	  \multicolumn{2}{>{\raggedright\arraybackslash}p{0.797\columnwidth}@{}}{%
	  Система формирует и направляет пользователю уведомление о выданном
	  материале и установленном сроке возврата.} \\
	  \multicolumn{3}{@{}>{\raggedright\arraybackslash}p{\columnwidth}@{}}{%
	  \textbf{Альтернативный сценарий: У пользователя есть просроченные
	  материалы (шаг №4)}} \\
	  4.А.1 &
	  \multicolumn{2}{>{\raggedright\arraybackslash}p{0.797\columnwidth}@{}}{%
	  Система обнаруживает у пользователя просроченные материалы.} \\
	  4.А.2 &
	  \multicolumn{2}{>{\raggedright\arraybackslash}p{0.797\columnwidth}@{}}{%
	  Система блокирует операцию выдачи нового материала.} \\
	  4.А.3 &
	  \multicolumn{2}{>{\raggedright\arraybackslash}p{0.797\columnwidth}@{}}{%
	  Система информирует библиотекаря и пользователя о причине блокировки
	  выдачи.} \\
	  4.А.4 &
	  \multicolumn{2}{>{\raggedright\arraybackslash}p{0.797\columnwidth}@{}}{%
	  Альтернативный сценарий завершается без продолжения основного
	  сценария.} \\
	  \multicolumn{3}{@{}>{\raggedright\arraybackslash}p{\columnwidth}@{}}{%
	  \textbf{Альтернативный сценарий: Материал зарезервирован другим
	  пользователем (шаг №3)}} \\
	  3.B.1 &
	  \multicolumn{2}{>{\raggedright\arraybackslash}p{0.797\columnwidth}@{}}{%
	  Система определяет, что выбранный материал зарезервирован другим
	  пользователем.} \\
	  3.B.2 &
	  \multicolumn{2}{>{\raggedright\arraybackslash}p{0.797\columnwidth}@{}}{%
	  Система запрещает выдачу материала текущему пользователю.} \\
	  3.B.3 &
	  \multicolumn{2}{>{\raggedright\arraybackslash}p{0.797\columnwidth}@{}}{%
	  Система информирует библиотекаря и пользователя о невозможности выдачи
	  материала.} \\
	  3.B.4 &
	  \multicolumn{2}{>{\raggedright\arraybackslash}p{0.797\columnwidth}@{}}{%
	  Альтернативный сценарий завершается без продолжения основного
	  сценария.} \\
	  \multicolumn{3}{@{}>{\raggedright\arraybackslash}p{\columnwidth}@{}}{%
	  \textbf{Альтернативный сценарий: Изменение срока выдачи библиотекарем
	  (шаг №5)}} \\
	  5.C.1 &
	  \multicolumn{2}{>{\raggedright\arraybackslash}p{0.797\columnwidth}@{}}{%
	  Библиотекарь принимает решение изменить срок выдачи материала
	  пользователю в соответствии с внутренними правилами библиотеки.} \\
	  5.C.2 &
	  \multicolumn{2}{>{\raggedright\arraybackslash}p{0.797\columnwidth}@{}}{%
	  Библиотекарь задаёт новый срок возврата материала.} \\
	  5.C.3 &
	  \multicolumn{2}{>{\raggedright\arraybackslash}p{0.797\columnwidth}@{}}{%
	  Система обновляет срок возврата материала в записи выдачи.} \\
	  5.C.4 &
	  \multicolumn{2}{>{\raggedright\arraybackslash}p{0.797\columnwidth}@{}}{%
	  Система при необходимости формирует и направляет пользователю актуальное
	  уведомление о сроке возврата.} \\
	  5.C.5 &
	  \multicolumn{2}{>{\raggedright\arraybackslash}p{0.797\columnwidth}@{}}{%
	  Альтернативный сценарий завершается без возврата к основному
	  сценарию.} \\
	  \multicolumn{3}{@{}>{\raggedright\arraybackslash}p{\columnwidth}@{}}{%
	  \textbf{Исключения (Ошибки, отсутствие данных для отображения)}} \\
	  E1 & Сбой записи в базу данных при создании выдачи. & Выдача не
	  создана. \\
	  E2 & Ошибка синхронизации с удалённым филиалом. & Выдача не
	  подтверждена. \\
	  \textbf{Постусловия} &
	  \multicolumn{2}{>{\raggedright\arraybackslash}p{0.797\columnwidth}@{}}{%
	  Материал числится как выданный пользователю.} \\
	  \bottomrule
	  \end{longtable}
	
	\subsubsection{Прецедент 5: Возврат материала}
	
	\begin{longtable}[]{@{}
		>{\raggedright\arraybackslash}p{0.192\columnwidth}
		>{\raggedright\arraybackslash}p{0.425\columnwidth}
		>{\raggedright\arraybackslash}p{0.383\columnwidth}@{}}
	\toprule
	\multicolumn{3}{@{}>{\raggedright\arraybackslash}p{\columnwidth}@{}}{%
	\begin{minipage}[b]{\linewidth}\raggedright
	Возврат материала
	\end{minipage}} \\
	\midrule
	\endhead
	\textbf{Описание} &
	\multicolumn{2}{>{\raggedright\arraybackslash}p{0.808\columnwidth}@{}}{%
	Пользователь возвращает материал, библиотекарь регистрирует возврат в системе.} \\
	\textbf{Предусловия} &
	\multicolumn{2}{>{\raggedright\arraybackslash}p{0.808\columnwidth}@{}}{%
	\begin{minipage}[t]{\linewidth}\raggedright
	\begin{enumerate}
	\def\labelenumi{\arabic{enumi}.}
	\item
		Материал числится за пользователем как выданный.
	\item
		Пользователь обращается в библиотеку для возврата материала.
	\end{enumerate}
	\end{minipage}} \\
	\textbf{Акторы (Участники)} &
	\multicolumn{2}{>{\raggedright\arraybackslash}p{0.808\columnwidth}@{}}{%
	Пользователь, Система, Библиотекарь} \\
	\multicolumn{3}{@{}>{\raggedright\arraybackslash}p{\columnwidth}@{}}{%
	\textbf{Основной сценарий}} \\
	\textbf{№ шага} &
	\multicolumn{2}{>{\raggedright\arraybackslash}p{0.808\columnwidth}@{}}{%
	\textbf{Действие}} \\
	1 &
	\multicolumn{2}{>{\raggedright\arraybackslash}p{0.808\columnwidth}@{}}{%
	Библиотекарь находит в системе запись о выдаче возвращаемого материала конкретному пользователю.} \\
	2 &
	\multicolumn{2}{>{\raggedright\arraybackslash}p{0.808\columnwidth}@{}}{%
	Библиотекарь регистрирует факт возврата материала в системе.} \\
	3 &
	\multicolumn{2}{>{\raggedright\arraybackslash}p{0.808\columnwidth}@{}}{%
	Система определяет фактическую дату возврата материала.} \\
	4 &
	\multicolumn{2}{>{\raggedright\arraybackslash}p{0.808\columnwidth}@{}}{%
	Система сравнивает фактическую дату возврата с установленным сроком возврата.} \\
	5 &
	\multicolumn{2}{>{\raggedright\arraybackslash}p{0.808\columnwidth}@{}}{%
	При отсутствии просрочки система изменяет статус материала на «доступен».} \\
	6 &
	\multicolumn{2}{>{\raggedright\arraybackslash}p{0.808\columnwidth}@{}}{%
	При наличии просрочки система рассчитывает размер штрафа в соответствии с действующими тарифами.} \\
	7 &
	\multicolumn{2}{>{\raggedright\arraybackslash}p{0.808\columnwidth}@{}}{%
	Система формирует и направляет пользователю уведомление о результатах возврата материала.} \\
	8 &
	\multicolumn{2}{>{\raggedright\arraybackslash}p{0.808\columnwidth}@{}}{%
	При наличии рассчитанного штрафа система формирует и направляет пользователю уведомление о начисленном штрафе.} \\
	\multicolumn{3}{@{}>{\raggedright\arraybackslash}p{\columnwidth}@{}}{%
	\textbf{Альтернативный сценарий: Возврат повреждённого материала (шаг №2)}} \\
	2.А.1 &
	\multicolumn{2}{>{\raggedright\arraybackslash}p{0.808\columnwidth}@{}}{%
	Библиотекарь при приёме материала обнаруживает повреждение экземпляра.} \\
	2.А.2 &
	\multicolumn{2}{>{\raggedright\arraybackslash}p{0.808\columnwidth}@{}}{%
	Библиотекарь фиксирует состояние повреждённого экземпляра в системе.} \\
	2.А.3 &
	\multicolumn{2}{>{\raggedright\arraybackslash}p{0.808\columnwidth}@{}}{%
	Библиотекарь инициирует создание акта о повреждении материала.} \\
	2.А.4 &
	\multicolumn{2}{>{\raggedright\arraybackslash}p{0.808\columnwidth}@{}}{%
	Система сохраняет сведения об акте повреждения материала.} \\
	2.А.5 &
	\multicolumn{2}{>{\raggedright\arraybackslash}p{0.808\columnwidth}@{}}{%
	При необходимости система инициирует процедуру расчёта компенсации или иного вида ответственности в соответствии с правилами библиотеки.} \\
	2.А.6 &
	\multicolumn{2}{>{\raggedright\arraybackslash}p{0.808\columnwidth}@{}}{%
	После фиксации повреждения возврат материала продолжает обрабатываться по основному сценарию с шага 3 ИЛИ переходит к отдельной процедуре урегулирования (в зависимости от принятых правил).} \\
	\multicolumn{3}{@{}>{\raggedright\arraybackslash}p{\columnwidth}@{}}{%
	\textbf{Альтернативный сценарий: Досрочный возврат материала (шаг №4)}} \\
	4.B.1 &
	\multicolumn{2}{>{\raggedright\arraybackslash}p{0.808\columnwidth}@{}}{%
	Пользователь возвращает материал до наступления срока возврата.} \\
	4.B.2 &
	\multicolumn{2}{>{\raggedright\arraybackslash}p{0.808\columnwidth}@{}}{%
	Система фиксирует отсутствие просрочки при выполнении сравнения дат возврата.} \\
	4.B.3 &
	\multicolumn{2}{>{\raggedright\arraybackslash}p{0.808\columnwidth}@{}}{%
	Система пропускает расчёт штрафа и изменяет статус материала на «доступен».} \\
	4.B.4 &
	\multicolumn{2}{>{\raggedright\arraybackslash}p{0.808\columnwidth}@{}}{%
	Альтернативный сценарий завершается продолжением основного сценария с шага 7.} \\
	\multicolumn{3}{@{}>{\raggedright\arraybackslash}p{\columnwidth}@{}}{%
	\textbf{Альтернативный сценарий: Малая просрочка и отмена штрафа (шаг №6)}} \\
	6.C.1 &
	\multicolumn{2}{>{\raggedright\arraybackslash}p{0.808\columnwidth}@{}}{%
	Система определяет, что просрочка возврата материала не превышает установленный порог (например, одни сутки).} \\
	6.C.2 &
	\multicolumn{2}{>{\raggedright\arraybackslash}p{0.808\columnwidth}@{}}{%
	Система рассчитывает штраф в соответствии с тарифами.} \\
	6.C.3 &
	\multicolumn{2}{>{\raggedright\arraybackslash}p{0.808\columnwidth}@{}}{%
	Библиотекарь принимает решение отменить начисленный штраф при малой просрочке.} \\
	6.C.4 &
	\multicolumn{2}{>{\raggedright\arraybackslash}p{0.808\columnwidth}@{}}{%
	Библиотекарь инициирует в системе отмену рассчитанного штрафа.} \\
	6.C.5 &
	\multicolumn{2}{>{\raggedright\arraybackslash}p{0.808\columnwidth}@{}}{%
	Система отменяет начисленный штраф и сохраняет соответствующую запись.} \\
	6.C.6 &
	\multicolumn{2}{>{\raggedright\arraybackslash}p{0.808\columnwidth}@{}}{%
	Система формирует и направляет пользователю уведомление об отмене штрафа.} \\
	6.C.7 &
	\multicolumn{2}{>{\raggedright\arraybackslash}p{0.808\columnwidth}@{}}{%
	Альтернативный сценарий завершается без возврата к основному сценарию.} \\
	\multicolumn{3}{@{}>{\raggedright\arraybackslash}p{\columnwidth}@{}}{%
	\textbf{Исключения (Ошибки, отсутствие данных для отображения)}} \\
	E1 & Ошибка обновления статуса материала. & Статус материала не обновлён. \\
	E2 & Ошибка расчёта штрафа из-за отсутствия данных о тарифе. & Отсутствуют данные о тарифе для расчёта штрафа. \\
	\textbf{Постусловия} &
	\multicolumn{2}{>{\raggedright\arraybackslash}p{0.808\columnwidth}@{}}{%
	Материал возвращён и снова доступен. При необходимости начислен штраф.} \\
	\bottomrule
	\end{longtable}
	
	\subsubsection{Прецедент 6: Погашение штрафа}
	
	\begin{longtable}[]{@{}
		>{\raggedright\arraybackslash}p{0.207\columnwidth}
		>{\raggedright\arraybackslash}p{0.339\columnwidth}
		>{\raggedright\arraybackslash}p{0.454\columnwidth}@{}}
	\toprule
	\multicolumn{3}{@{}>{\raggedright\arraybackslash}p{\columnwidth}@{}}{%
	\begin{minipage}[b]{\linewidth}\raggedright
	Погашение штрафа
	\end{minipage}} \\
	\midrule
	\endhead
	\textbf{Описание} &
	\multicolumn{2}{>{\raggedright\arraybackslash}p{0.793\columnwidth}@{}}{%
	Пользователь оплачивает начисленный штраф через платёжный сервис.} \\
	\textbf{Предусловия} &
	\multicolumn{2}{>{\raggedright\arraybackslash}p{0.793\columnwidth}@{}}{%
	\begin{minipage}[t]{\linewidth}\raggedright
	\begin{enumerate}
	\def\labelenumi{\arabic{enumi}.}
	\item
		Пользователь авторизован.
	\item
		В системе существует хотя бы один активный штраф, начисленный пользователю.
	\end{enumerate}
	\end{minipage}} \\
	\textbf{Акторы (Участники)} &
	\multicolumn{2}{>{\raggedright\arraybackslash}p{0.793\columnwidth}@{}}{%
	Пользователь, Система} \\
	\multicolumn{3}{@{}>{\raggedright\arraybackslash}p{\columnwidth}@{}}{%
	\textbf{Основной сценарий}} \\
	\textbf{№ шага} &
	\multicolumn{2}{>{\raggedright\arraybackslash}p{0.793\columnwidth}@{}}{%
	\textbf{Действие}} \\
	1 &
	\multicolumn{2}{>{\raggedright\arraybackslash}p{0.793\columnwidth}@{}}{%
	Пользователь запрашивает в системе список своих активных штрафов.} \\
	2 &
	\multicolumn{2}{>{\raggedright\arraybackslash}p{0.793\columnwidth}@{}}{%
	Система формирует список активных штрафов пользователя.} \\
	3 &
	\multicolumn{2}{>{\raggedright\arraybackslash}p{0.793\columnwidth}@{}}{%
	Система отображает пользователю список активных штрафов.} \\
	4 &
	\multicolumn{2}{>{\raggedright\arraybackslash}p{0.793\columnwidth}@{}}{%
	Пользователь выбирает штраф для оплаты.} \\
	5 &
	\multicolumn{2}{>{\raggedright\arraybackslash}p{0.793\columnwidth}@{}}{%
	Пользователь инициирует оплату выбранного штрафа.} \\
	6 &
	\multicolumn{2}{>{\raggedright\arraybackslash}p{0.793\columnwidth}@{}}{%
	Система передаёт параметры выбранного штрафа в платёжный сервис и передаёт ему управление для проведения операции.} \\
	7 &
	\multicolumn{2}{>{\raggedright\arraybackslash}p{0.793\columnwidth}@{}}{%
	Пользователь вводит платёжные реквизиты в платёжном сервисе.} \\
	8 &
	\multicolumn{2}{>{\raggedright\arraybackslash}p{0.793\columnwidth}@{}}{%
	Пользователь подтверждает выполнение платежа.} \\
	9 &
	\multicolumn{2}{>{\raggedright\arraybackslash}p{0.793\columnwidth}@{}}{%
	Платёжный сервис обрабатывает платёж.} \\
	10 &
	\multicolumn{2}{>{\raggedright\arraybackslash}p{0.793\columnwidth}@{}}{%
	Платёжный сервис формирует результат обработки платежа.} \\
	11 &
	\multicolumn{2}{>{\raggedright\arraybackslash}p{0.793\columnwidth}@{}}{%
	Платёжный сервис передаёт результат обработки платежа в систему.} \\
	12 &
	\multicolumn{2}{>{\raggedright\arraybackslash}p{0.793\columnwidth}@{}}{%
	При успешной оплате система обновляет статус штрафа на «оплачен».} \\
	13 &
	\multicolumn{2}{>{\raggedright\arraybackslash}p{0.793\columnwidth}@{}}{%
	Система формирует и направляет пользователю уведомление об успешной оплате штрафа.} \\
	\multicolumn{3}{@{}>{\raggedright\arraybackslash}p{\columnwidth}@{}}{%
	\textbf{Альтернативный сценарий: Отмена оплаты пользователем (шаг №8)}} \\
	8.А.1 &
	\multicolumn{2}{>{\raggedright\arraybackslash}p{0.793\columnwidth}@{}}{%
	Пользователь отменяет подтверждение платежа в платёжном сервисе.} \\
	8.А.2 &
	\multicolumn{2}{>{\raggedright\arraybackslash}p{0.793\columnwidth}@{}}{%
	Платёжный сервис прекращает обработку операции оплаты.} \\
	8.А.3 &
	\multicolumn{2}{>{\raggedright\arraybackslash}p{0.793\columnwidth}@{}}{%
	Платёжный сервис передаёт в систему информацию об отменённой операции.} \\
	8.А.4 &
	\multicolumn{2}{>{\raggedright\arraybackslash}p{0.793\columnwidth}@{}}{%
	Система сохраняет исходный статус штрафа как «не оплачен».} \\
	8.А.5 &
	\multicolumn{2}{>{\raggedright\arraybackslash}p{0.793\columnwidth}@{}}{%
	Система возвращает пользователя к просмотру списка активных штрафов (аналог состояния после шага 3 основного сценария).} \\
	8.А.6 &
	\multicolumn{2}{>{\raggedright\arraybackslash}p{0.793\columnwidth}@{}}{%
	Альтернативный сценарий завершается продолжением основного сценария с шага 3.} \\
	\multicolumn{3}{@{}>{\raggedright\arraybackslash}p{\columnwidth}@{}}{%
	\textbf{Альтернативный сценарий: Платёж отклонён банком (шаг №9--11)}} \\
	9.B.1 &
	\multicolumn{2}{>{\raggedright\arraybackslash}p{0.793\columnwidth}@{}}{%
	Платёжный сервис получает от банка отказ в проведении операции оплаты.} \\
	9.B.2 &
	\multicolumn{2}{>{\raggedright\arraybackslash}p{0.793\columnwidth}@{}}{%
	Платёжный сервис формирует информацию о неуспешной операции.} \\
	9.B.3 &
	\multicolumn{2}{>{\raggedright\arraybackslash}p{0.793\columnwidth}@{}}{%
	Платёжный сервис передаёт в систему информацию об отказе в проведении платежа.} \\
	9.B.4 &
	\multicolumn{2}{>{\raggedright\arraybackslash}p{0.793\columnwidth}@{}}{%
	Система сохраняет статус штрафа как «не оплачен».} \\
	9.B.5 &
	\multicolumn{2}{>{\raggedright\arraybackslash}p{0.793\columnwidth}@{}}{%
	Система информирует пользователя о том, что платёж отклонён.} \\
	9.B.6 &
	\multicolumn{2}{>{\raggedright\arraybackslash}p{0.793\columnwidth}@{}}{%
	Система предлагает пользователю повторить попытку оплаты штрафа.} \\
	9.B.7 &
	\multicolumn{2}{>{\raggedright\arraybackslash}p{0.793\columnwidth}@{}}{%
	Альтернативный сценарий завершается, пользователь при желании может повторно инициировать основной сценарий.} \\
	\multicolumn{3}{@{}>{\raggedright\arraybackslash}p{\columnwidth}@{}}{%
	\textbf{Альтернативный сценарий: Задержка подтверждения от платёжного сервиса (шаг №8)}} \\
	8.C.1 &
	\multicolumn{2}{>{\raggedright\arraybackslash}p{0.793\columnwidth}@{}}{%
	Система фиксирует отсутствие результата от платёжного сервиса в установленный интервал времени.} \\
	8.C.2 &
	\multicolumn{2}{>{\raggedright\arraybackslash}p{0.793\columnwidth}@{}}{%
	Система устанавливает для штрафа промежуточный статус (например, «ожидает подтверждения оплаты»).} \\
	8.C.3 &
	\multicolumn{2}{>{\raggedright\arraybackslash}p{0.793\columnwidth}@{}}{%
	Система продолжает периодически запрашивать у платёжного сервиса состояние операции.} \\
	8.C.4 &
	\multicolumn{2}{>{\raggedright\arraybackslash}p{0.793\columnwidth}@{}}{%
	При получении подтверждения успешной оплаты от платёжного сервиса система обновляет статус штрафа на «оплачен».} \\
	8.C.5 &
	\multicolumn{2}{>{\raggedright\arraybackslash}p{0.793\columnwidth}@{}}{%
	Система формирует и направляет пользователю уведомление об успешной оплате штрафа.} \\
	8.C.6 &
	\multicolumn{2}{>{\raggedright\arraybackslash}p{0.793\columnwidth}@{}}{%
	При получении информации об отказе система действует по шагам альтернативного сценария B.} \\
	8.C.7 &
	\multicolumn{2}{>{\raggedright\arraybackslash}p{0.793\columnwidth}@{}}{%
	Альтернативный сценарий завершается.} \\
	\multicolumn{3}{@{}>{\raggedright\arraybackslash}p{\columnwidth}@{}}{%
	\textbf{Исключения (Ошибки, отсутствие данных для отображения)}} \\
	E1 & Ошибка соединения с платёжным сервисом. & Система уведомляет о невозможности проведения операции. \\
	E2 & Сбой записи данных о транзакции. & Система не меняет статус штрафа и логирует ошибку. \\
	\textbf{Постусловия} &
	\multicolumn{2}{>{\raggedright\arraybackslash}p{0.793\columnwidth}@{}}{%
	Штраф оплачен и удалён из списка активных.} \\
	\bottomrule
	\end{longtable}
	
	\subsubsection{Функциональные требования}
	
	На основе детализированных прецедентов использования системы сформулированы функциональные требования, которые определяют конкретные возможности системы.
	
	\begin{enumerate}
		\item \textbf{Регистрация / Авторизация пользователя}
		\begin{enumerate}
			\item Система должна проверять корректность введённых пользователем данных регистрации (e-mail, пароль, подтверждение пароля, ФИО).
			\item Система должна проверять уникальность указанного пользователем e-mail при регистрации.
			\item Система должна создавать учётную запись пользователя при успешной проверке регистрационных данных.
			\item Система должна отправлять пользователю письмо с подтверждением регистрации на указанный e-mail.
			\item Система должна активировать учётную запись пользователя при переходе по ссылке из письма подтверждения.
			\item Система должна отказывать в регистрации, если указанный пользователем e-mail уже используется другой учётной записью.
			\item Система должна отображать пользователю сообщение о несоответствии пароля и его подтверждения при их различии.
			\item Система должна позволять пользователю повторно отправлять письмо активации учётной записи.
			\item Система должна сообщать пользователю об ошибках соединения или сбоях базы данных при выполнении операций регистрации и авторизации.
			\item Система должна удалять неактивированные учётные записи пользователей по истечении установленного срока.
			\item Система должна позволять пользователю, имеющему активную учётную запись, выполнять вход в систему.
			\item Система должна блокировать невалидные сессии и предотвращать доступ при некорректных попытках авторизации.
		\end{enumerate}
		
		\item \textbf{Поиск материалов}
		\begin{enumerate}
			\item Система должна предоставлять пользователю средства для ввода поисковых запросов по каталогу материалов.
			\item Система должна искать материалы по ключевым характеристикам (например, названию, автору, идентификатору, ISBN и др.).
			\item Система должна отображать пользователю список найденных материалов с основной информацией о каждом материале.
			\item Система должна поддерживать фильтрацию результатов поиска по заданным пользователем параметрам (жанру, году издания, типу материала, наличию и т. п.).
			\item Система должна поддерживать сортировку результатов поиска (например, по алфавиту, дате издания, дате добавления).
			\item Система должна сообщать пользователю о том, что по заданным критериям не найдено ни одного материала.
			\item Система должна отображать пользователю список популярных или рекомендованных материалов при отсутствии введённого поискового запроса.
			\item Система должна отображать пользователю сообщение о некорректной комбинации фильтров, если заданные параметры отбора не дают результатов.
			\item Система должна обеспечивать корректную работу поиска по каталогу при высокой нагрузке и большом объёме данных.
			\item Система должна сообщать пользователю о временной недоступности поиска при сбое соединения или внутренних ошибках.
		\end{enumerate}
		
		\item \textbf{Оформление брони}
		\begin{enumerate}
			\item Система должна позволять авторизованному пользователю инициировать бронирование выбранного материала из каталога.
			\item Система должна проверять доступность экземпляров материала для бронирования перед созданием записи брони.
			\item Система должна создавать запись бронирования материала за пользователем при успешной проверке условий.
			\item Система должна уведомлять библиотекаря о созданной пользователем брони.
			\item Система должна уведомлять пользователя об успешном создании бронирования материала.
			\item Система должна предлагать пользователю включиться в очередь ожидания, если все экземпляры выбранного материала уже забронированы.
			\item Система должна отказывать пользователю в создании новой брони, если превышен установленный лимит активных бронирований.
			\item Система должна позволять пользователю отменять созданную бронь до её подтверждения или обработки библиотекарем.
			\item Система должна сохранять все операции, связанные с бронированием материалов, в журнале действий.
			\item Система должна обрабатывать ошибки записи в базу данных при создании или отмене брони и уведомлять пользователя о невозможности выполнения операции.
			\item Система должна предотвращать дублирование активных броней одного и того же материала для одного пользователя.
		\end{enumerate}
		
		\item \textbf{Оформление выдачи материала}
		\begin{enumerate}
			\item Система должна позволять библиотекарю находить пользователя по идентификатору, имени или иному уникальному атрибуту при оформлении выдачи.
			\item Система должна проверять отсутствие у пользователя активных блокирующих штрафов или критических просрочек перед выдачей материала.
			\item Система должна проверять доступность выбранного экземпляра материала для выдачи пользователю.
			\item Система должна фиксировать факт выдачи материала пользователю с указанием даты выдачи, срока возврата и места (филиала) выдачи.
			\item Система должна уведомлять пользователя о выданных материалах и установленных сроках возврата.
			\item Система должна блокировать выдачу новых материалов пользователю при наличии у него просроченных материалов.
			\item Система должна предотвращать выдачу материала пользователю, если данный экземпляр зарезервирован другим пользователем.
			\item Система должна позволять библиотекарю изменять срок возврата материала в соответствии с внутренними правилами библиотеки.
			\item Система должна обрабатывать ошибки записи данных о выдаче в базе данных и уведомлять библиотекаря о невозможности завершения операции.
			\item Система должна обеспечивать синхронизацию данных о выдаче материалов между филиалами библиотеки при распределённом хранении данных.
			\item Система должна хранить журнал всех выданных экземпляров с указанием дат выдачи, возврата и идентификаторов пользователей и материалов.
		\end{enumerate}
		
		\item \textbf{Возврат материала}
		\begin{enumerate}
			\item Система должна позволять библиотекарю находить запись выдачи по пользователю, материалу или идентификатору экземпляра при возврате.
			\item Система должна фиксировать факт возврата материала в учётной записи выдачи.
			\item Система должна проверять фактическую дату возврата материала и сравнивать её с установленным сроком возврата.
			\item Система должна начислять пользователю штраф при обнаружении просрочки возврата в соответствии с действующими тарифами.
			\item Система должна позволять библиотекарю вручную отменять начисленный штраф при малой просрочке в рамках установленных правил.
			\item Система должна изменять статус возвращённого материала на «доступен» после успешной регистрации возврата.
			\item Система должна отправлять пользователю уведомление о принятом возврате материала.
			\item Система должна уведомлять пользователя о начисленном штрафе при наличии просрочки возврата.
			\item Система должна позволять библиотекарю фиксировать факт повреждения материала при его возврате.
			\item Система должна создавать запись (акт) о повреждении материала с указанием состояния экземпляра и необходимых деталей.
			\item Система должна обрабатывать ошибки при обновлении статуса материала и начислении штрафа, не допуская неконсистентного состояния данных.
			\item Система должна обрабатывать случаи отсутствия данных о тарифах при расчёте штрафа и уведомлять об этом библиотекаря или администратора.
		\end{enumerate}
		
		\item \textbf{Погашение штрафа}
		\begin{enumerate}
			\item Система должна предоставлять пользователю доступ к актуальному списку его активных штрафов.
			\item Система должна позволять пользователю выбирать штраф из списка и инициировать его оплату.
			\item Система должна перенаправлять пользователя для оплаты штрафа в защищённый платёжный сервис.
			\item Система должна получать от платёжного сервиса статус выполненной транзакции (успех, отказ, отмена, ожидание подтверждения).
			\item Система должна обновлять статус штрафа на «оплачен» при получении успешного подтверждения оплаты от платёжного сервиса.
			\item Система должна отображать пользователю сообщение об успешной оплате штрафа после изменения его статуса.
			\item Система должна не изменять статус штрафа при отмене пользователем операции оплаты или при отказе в проведении платежа.
			\item Система должна повторно запрашивать статус операции оплаты у платёжного сервиса при задержке подтверждения результата.
			\item Система должна обрабатывать ошибки соединения с платёжным сервисом и информировать пользователя о невозможности завершения операции оплаты.
			\item Система должна обрабатывать ошибки записи данных о транзакции оплаты в базу данных и обеспечивать целостность статусов штрафов.
			\item Система должна обеспечивать журналирование всех операций, связанных с оплатой штрафов, для последующего аудита.
			\item Система должна защищать передаваемые платёжные данные пользователя от несанкционированного доступа и утечек.
		\end{enumerate}
	\end{enumerate}
	
	\subsection{Итерация 4: Диаграммы архитектуры, классов и последовательностей}
	
	\subsubsection{Диаграмма контекста (C1)}
	
	Диаграмма контекста определяет границы системы и её взаимодействие с внешними акторами и системами.
	
	\begin{figure}[H]
		\centering
		\includegraphics[width=0.6\textwidth]{Lab4_sa_sd/C1-diagram.png}
		\caption{Диаграмма контекста (Context Diagram - C1).}
		\label{fig:c1}
	\end{figure}
	
	\subsubsection{Диаграмма контейнеров (C2)}
	
	Диаграмма контейнеров показывает высокоуровневую архитектуру системы, включая основные приложения, базы данных и внешние системы.
	
	\begin{figure}[H]
		\centering
		\includegraphics[width=1\textwidth]{Lab4_sa_sd/C2-diagram.png}
		\caption{Диаграмма контейнеров (Container Diagram - C2).}
		\label{fig:c2}
	\end{figure}
	
	\subsubsection{Диаграмма компонентов (C3)}
	
	Диаграмма компонентов детализирует внутреннюю структуру одного из ключевых контейнеров системы, показывая основные компоненты и их взаимодействие.
	
	\begin{figure}[H]
		\centering
		\includegraphics[width=1\textwidth]{Lab4_sa_sd/C3-diagram.png}
		\caption{Диаграмма компонентов (Component Diagram - C3) для одного ключевого контейнера.}
		\label{fig:c3}
	\end{figure}
	
	\subsubsection{Диаграмма классов (C4)}
	
	Диаграмма классов описывает структуру данных системы, основные сущности и их взаимосвязи.
	
	\begin{figure}[H]
		\centering
		\includegraphics[width=1\textwidth]{Lab4_sa_sd/C4-diagram.png}
		\caption{Диаграмма классов (Class Diagram - C4).}
		\label{fig:c4}
	\end{figure}
	
	\subsubsection{Диаграмма последовательностей}
	
	Диаграмма последовательностей показывает динамическое взаимодействие объектов в рамках одного из ключевых сценариев использования системы.
	
	\begin{figure}[H]
		\centering
		\includegraphics[width=0.7\textwidth]{Lab4_sa_sd/Sequence Diagram.png}
		\caption{Диаграмма последовательностей (Sequence Diagram).}
		\label{fig:sequence}
	\end{figure}
	
	\section{Нефункциональные требования}
	
	Нефункциональные требования определяют качественные характеристики системы, которые должны быть обеспечены при её реализации. Требования сформулированы в измеримом виде для обеспечения возможности их проверки.
	
	\subsection{Производительность}
	\begin{enumerate}
		\item Система каталога должна обрабатывать не менее 95\% поисковых запросов по каталогу при нагрузке до 100 одновременных пользователей за время не более 1.5 секунд в 95\% измерений за календарные сутки.
		\item Система обработки операций выдачи должна выполнять регистрацию выдачи или возврата одной книги при нагрузке до 100 одновременных библиотекарей за время не более 2 секунд в 95\% измерений.
		\item Подсистема отчётности должна формировать отчёт о выданных книгах по одной библиотеке объёмом до 100\,000 записей за время не более 5 секунд при нормальной нагрузке.
		\item Подсистема продления должна регистрировать продление срока одной выдачи пользователем при нагрузке до 200 запросов в минуту за время не более 2 секунд.
		\item Подсистема управления каталогом должна сохранять изменения карточки одной книги (редактирование полей или добавление новой книги) при нагрузке до 50 одновременных библиотекарей за время не более 3 секунд.
		\item Подсистема поиска с фильтрами должна возвращать пользователю результаты поиска с применёнными фильтрами (автор, жанр, год издания) при объёме каталога до 100\,000 записей за время не более 2 секунд.
	\end{enumerate}
	
	\subsection{Надёжность и доступность}
	\begin{enumerate}
		\item Система в целом должна обеспечивать доступность (uptime) не менее 98\% в течение календарного месяца при эксплуатации в штатном режиме.
		\item Подсистема восстановления должна восстанавливать работоспособность системы после критического сбоя за время не более 2 часов с момента фиксации сбоя в системе мониторинга.
		\item Подсистема резервного копирования должна выполнять ежедневное полное резервное копирование базы данных объёмом до 10~ГБ в промежутке с 02{:}00 до 04{:}00 по серверному времени без остановки пользовательских операций более чем на 2 минуты.
		\item Подсистема восстановления данных должна восстанавливать базу данных из последней успешной резервной копии объёмом до 10~ГБ за время не более 1 часа.
	\end{enumerate}
	
	\subsection{Безопасность}
	\begin{enumerate}
		\item Подсистема аутентификации должна хранить пароли пользователей только в виде хэш-сумм, рассчитанных алгоритмом bcrypt с индивидуальной солью длиной не менее 16 байт во всех пользовательских записях.
		\item Подсистема авторизации должна разрешать доступ к операциям управления пользователями и каталогом материалов только учетным записям с ролями ``Библиотекарь'' или ``Администратор'' при проверке прав в каждой защищаемой операции.
		\item Транспортный уровень системы должен обеспечивать шифрование всех запросов и ответов между клиентом и сервером с использованием протокола TLS версии не ниже 1.2 во 100\% внешних соединений.
		\item Подсистема аудита должна записывать в журнал аудита все действия библиотекарей и администраторов (создание, изменение, удаление сущностей) с указанием идентификатора пользователя, времени и типа операции для 100\% таких действий.
		\item Подсистема защиты от перебора паролей должна блокировать учётную запись пользователя на 15 минут после 5 подряд неуспешных попыток входа, совершённых в интервале 15 минут.
		\item Подсистема подтверждения критических операций должна требовать явного подтверждения пользователя и записывать в журнал аудита каждую операцию удаления материалов, начисления штрафов и изменения статуса библиотек в 100\% случаев.
	\end{enumerate}
	
	\subsection{Удобство использования (Usability)}
	\begin{enumerate}
		\item Пользовательский веб-интерфейс должен позволять новому пользователю пройти регистрацию и авторизацию при наличии стабильного соединения за время не более 3 минут в 90\% сценариев юзабилити-тестирования.
		\item Пользовательский интерфейс должен позволять читателю оформить первую бронь или выдачу книги при знании названия или автора за время не более 4 минут в 90\% тестовых сценариев.
		\item Интерфейс личного кабинета должен позволять зарегистрированному читателю продлить срок выдачи книги не более чем в два клика мышью (или тапа) без перехода на дополнительные страницы.
		\item Интерфейс библиотекаря должен позволять добавить карточку новой книги с заполнением обязательных полей за время не более 1 минуты при наличии готовых данных в 95\% тестовых сценариев.
		\item Подсистема поиска должна обеспечивать, чтобы не менее 90\% книг, считающихся релевантными экспертами по введённому ключевому слову, отображались на первой странице результатов поиска.
		\item Подсистема уведомлений должна доставлять электронное уведомление о наступлении просрочки возврата книги пользователю в течение не более 1 часа после наступления даты возврата при доступности внешнего сервиса уведомлений.
	\end{enumerate}
	
	\subsection{Масштабируемость}
	\begin{enumerate}
		\item Система приложений должна поддерживать одновременную работу не менее 200 активных пользователей при среднем количестве 3 запросов в минуту от одного пользователя без увеличения среднего времени ответа основных операций более чем на 20\% относительно базового уровня.
		\item Подсистема подключения библиотек должна позволять добавить одну новую библиотеку с заполнением конфигурационных данных и активацией в системе за время не более 24 часов без полной остановки сервиса.
		\item Подсистема хранения данных должна поддерживать объём каталога до 100\,000 записей материалов без роста среднего времени выполнения поискового запроса более чем до 2 секунд.
		\item Система обработки операций выдачи и возврата должна выдерживать до 50 одновременных операций выдачи/возврата в секунду без снижения средней скорости ответа поиска по каталогу более чем на 15\%.
		\item Инфраструктурный слой системы должен автоматически масштабироваться горизонтально (добавление экземпляров приложений) при среднем использовании CPU выше 80\% в течение 5 минут, снижая использование до уровня ниже 70\% в течение последующих 10 минут.
		\item Архитектура приложения должна обеспечивать возможность добавления новых функциональных модулей (например, поддержки новых типов материалов) путём развертывания отдельных сервисов без остановки существующих модулей более чем на 5 минут для одной библиотеки.
	\end{enumerate}
	
	\subsection{Поддерживаемость}
	\begin{enumerate}
		\item Подсистема документации должна публиковать обновлённую пользовательскую и административную документацию в онлайн-доступе не позднее 3 рабочих дней после утверждения изменения функциональных требований.
		\item Процесс внедрения новой функциональности в промышленное окружение должен занимать не более 2 недель с момента утверждения спецификации в 90\% релизов.
		\item Подсистема мониторинга должна отображать информацию о критических ошибках и сбоях всех сервисов в админ-панели с задержкой не более 1 минуты от момента возникновения события.
		\item Подсистема аудита должна фиксировать любые изменения данных библиотекарем (создание, изменение, удаление материалов и читателей) в журнале аудита в 100\% случаев, обеспечивая возможность отката на уровне базы данных.
		\item Процесс модификации существующего модуля и его развёртывания в тестовом окружении не должен приводить к полной недоступности тестовой среды более чем на 1 час за один релиз.
		\item Система автоматического тестирования должна обеспечивать покрытие автотестами не менее 80\% критических пользовательских сценариев (поиск, бронирование, выдача, возврат, оплата штрафов), измеряемое метрикой покрытия по строкам кода и сценариям.
	\end{enumerate}
	
	\section{Заключение}
	
	В рамках данного проекта была проведена полная работа по системному анализу и проектированию информационной системы управления библиотекой. В результате выполненной работы были достигнуты следующие результаты:
	
	\begin{itemize}
		\item Проведён анализ стейкхолдеров системы и разработан план выявления требований, что позволило определить ключевых участников проекта и их интересы.
		
		\item Сформулированы бизнес-требования в формате SMART, включающие конкретные измеримые цели по увеличению эффективности работы библиотеки.
		
		\item Разработаны BPMN-диаграммы ключевых бизнес-процессов (выдача и возврат книг), которые наглядно демонстрируют логику работы системы.
		
		\item Создан Use Case Diagram, описывающий основные сценарии использования системы, и детализированы ключевые прецеденты использования с описанием основных и альтернативных сценариев.
		
		\item Разработана архитектура системы, включающая диаграммы контекста (C1), контейнеров (C2), компонентов (C3), классов (C4) и последовательностей, что обеспечивает полное понимание структуры и взаимодействия компонентов системы.
		
		\item Определены нефункциональные требования к системе, включающие требования к производительности, надёжности, безопасности, удобству использования, масштабируемости и поддерживаемости.
	\end{itemize}
	
	Разработанная система управления библиотекой решает поставленные задачи по автоматизации процессов выдачи и возврата книг, управления каталогом библиотечных материалов, бронирования и продления сроков выдачи. Архитектура системы обеспечивает масштабируемость, надёжность и безопасность, что позволяет эффективно использовать систему в условиях реальной эксплуатации.
	
	Все требования и архитектурные решения документированы и готовы к использованию на этапе разработки и внедрения системы.
	
\end{document}

