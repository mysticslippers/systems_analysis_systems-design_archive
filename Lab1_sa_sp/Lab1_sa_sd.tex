\documentclass[12pt]{report}

\usepackage{cmap}
\usepackage[T1,T2A]{fontenc}
\usepackage[utf8]{inputenc}
\usepackage[english, russian]{babel}
\usepackage{amssymb}
\usepackage{amsmath}
\usepackage{amsthm}
\usepackage{dsfont}
\usepackage{bm}
\usepackage{diagbox}
\usepackage{array}
\usepackage{placeins}
\usepackage[left=20mm,right=10mm,top=20mm,bottom=20mm,bindingoffset=2mm]{geometry}
\usepackage{indentfirst}
\usepackage[utf8]{inputenc}
\usepackage{float}
\usepackage[hidelinks]{hyperref}
\usepackage{graphicx}
\usepackage{xcolor}
\usepackage{listings}
\usepackage{minted}

\DeclareMathOperator{\N}{\mathbb{N}}
\DeclareMathOperator{\R}{\mathbb{R}}
\DeclareMathOperator{\Z}{\mathbb{Z}}
\DeclareMathOperator{\CC}{\mathbb{C}}
\DeclareMathOperator{\PP}{\mathrm{P}}
\DeclareMathOperator{\Expec}{\mathrm{E}}
\DeclareMathOperator{\Var}{\mathrm{Var}}
\DeclareMathOperator{\Cov}{\mathrm{Cov}}
\DeclareMathOperator{\asConv}{\xrightarrow{a.s.}}
\DeclareMathOperator{\LpConv}{\xrightarrow{Lp}}
\DeclareMathOperator{\pConv}{\xrightarrow{p}}
\DeclareMathOperator{\dConv}{\xrightarrow{d}}

\hypersetup{
	colorlinks=true,
	linkcolor=blue,
	citecolor=blue,
	urlcolor=blue
}

\lstset{language=Python, extendedchars=\true}

\lstdefinestyle{pythonstyle}{
	language=Python,
	backgroundcolor=\color{lightgray},
	commentstyle=\color{green},
	keywordstyle=\color{blue},
	stringstyle=\color{red},
	basicstyle=\ttfamily,
	frame=single,
	breaklines=true
}

\addto\captionsrussian{\renewcommand{\refname}{Список использованных источников}}

\begin{document}
	
	\begin{titlepage}
		\begin{center}
			\large{Федеральное государственное автономное образовательное учреждение высшего образования <<Национальный исследовательский университет ИТМО>>}
		\end{center}
		
		\vspace{15em}
		
		\begin{center}
			\huge{\textbf{Проектная работа}} \\
			\large{По дисциплине <<Системный анализ и системное проектирование>>} \\
			\large{Итерация №1}
		\end{center}
		
		\vspace{2em}
		
		\begin{flushright}
			\textit{\large{Выполнили:}} \\
			\large{Студент группы P3306} \\
			\large{Михайлов Дмитрий} \\
			\large{Андреевич} \\
			
			\large{Студент группы P3312} \\
			\large{Малышев Никита} \\
			\large{Александрович} \\
			\textit{\large{Преподаватель:}} \\
			\large{Маркина Татьяна} \\
			\large{Анатольевна}
		\end{flushright}
		
		\vspace{2cm}
		
		\begin{figure}[h]
			\centering
			\includegraphics[width=0.5\linewidth]{image.png}
		\end{figure}
		
		\begin{center}
			Санкт-Петербург \\
			2025 год
		\end{center}
	\end{titlepage}
	
	\tableofcontents
	\newpage
	
	\addcontentsline{toc}{section}{Карта стейкхолдеров}
	\section*{Карта стейкхолдеров}
	
	\begin{table}[h]
		\hspace{-1.5cm}
		\begin{tabular}{|l|l|c|c|c|}
			\hline
			\textbf{Стейкхолдер} & \textbf{Роль} & \textbf{Внутренний/внешний} & \textbf{Влияние} & \textbf{Интерес} \\
			\hline
			Директор библиотеки & Заказчик, руководитель & Внутренний & Высокое & Высокий \\
			\hline
			Читатель (пользователь) & Пользователь системы & Внешний & Низкое & Высокий \\
			\hline
			Библиотекарь & Администратор, оператор & Внутренний & Среднее & Высокий \\
			\hline
			IT-администратор & Технический специалист & Внутренний & Среднее & Средний \\
			\hline
			Разработчик & Технический специалист & Внутренний & Высокое & Средний  \\
			\hline
			Бухгалтер & Финансовый специалист & Внутренний & Среднее & Средний  \\
			\hline
			Городская администрация & Регулятор / инвестор & Внешний & Высокое & Средний \\
			\hline
			Провайдер платежей & Сервис-партнёр & Внешний & Среднее & Низкий \\
			\hline
		\end{tabular}
		\caption{Стейкхолдеры проекта: роль, контекст взаимодействия и интересы.}
		\label{tab:stakeholders}
	\end{table}
	
	\addcontentsline{toc}{section}{План выявления требований}
	\section*{План выявления требований}
	
	\begin{table}[H]
		\centering
		\begin{tabular}{|>{\raggedright\arraybackslash}p{0.3\textwidth}|>{\raggedright\arraybackslash}p{0.3\textwidth}|>{\raggedright\arraybackslash}p{0.3\textwidth}|}
			\hline
			\textbf{Стейкхолдер} & \textbf{Метод} &	 \textbf{Цель опроса} \\
			\hline
			Директор библиотеки & Интервью & Выяснить стратегические цели сервиса, приоритеты по развитию, бюджетные ограничения, требования к отчетности и KPI, регуляторные и аудиторские требования \\
			\hline
			Читатель (пользователь) & Анкеты & Понять, какие функции и удобства нужны читателю \\
			\hline
			Библиотекарь & Анкеты & Выяснить операционные требования: процессы выдачи/возврата, учёт материалов, каталогизация, проверки доступа, уведомления, отчётность \\
			\hline
			IT-администратор & Интервью & Определить требования к инфраструктуре, безопасности, резервному копированию, доступности сервисов, мониторингу и обновлениям, интеграциям \\
			\hline
			Разработчик & Наблюдение & Понять требования к API, интеграциям, архитектуре, используемым технологиям, ограничениям по времени релиза и качеству кода \\
			\hline
			Бухгалтер & Анкеты & Понять требования к учёту, финансовым операциям, интеграции с бухгалтерскими системами, отчётности \\
			\hline
			Городская администрация & Интервью & Определить требования к соответствию регуляторным нормам, взаимодействию с госструктурами, доступности услуг, инфраструктурным требованиям и вопросам открытых данных \\
			\hline
			Провайдер платежей & Интервью / Наблюдение & Понять требования к интеграции платежной инфраструктуры \\
			\hline
		\end{tabular}
		\caption{План выявления требований.}
		\label{tab:requirements}
	\end{table}
	\newpage
	
	
	\addcontentsline{toc}{section}{Предварительное видение продукта}
	\section*{Предварительное видение продукта}
	
	\begin{itemize}
		\item У пользователей должна быть возможность искать книги по автору, названию, теме и ISBN, с выдачей доступной копии через систему резерва.
		\item Время ответа поискового запроса не более \texttt{2 секунд} для \textbf{95\%} запросов.
		\item Доля успешных резерваций не менее \textbf{90\%} от впервые отправленных запросов.
		\item Предоставлять ежемесячные дашборды по выдаче, продлениям, популярности жанров и периоду пиковых нагрузок.
	\end{itemize}
	
\end{document}
