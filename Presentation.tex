\documentclass[aspectratio=169,12pt]{beamer}
\usetheme{Madrid}
\usecolortheme{default}

\usepackage[utf8]{inputenc}
\usepackage[T2A]{fontenc}
\usepackage[english, russian]{babel}
\usepackage{graphicx}
\usepackage{booktabs}

\title{Система управления библиотекой}
\subtitle{Проектная работа по дисциплине <<Системный анализ и системное проектирование>>}
\author{Михайлов Дмитрий Андреевич (P3306) \\ Малышев Никита Александрович (P3312)}
\institute{Университет ИТМО}
\date{2025}

\begin{document}

% Слайд 1: Титульный
\begin{frame}
	\titlepage
\end{frame}

% Слайд 2: Проблема и Решение
\begin{frame}{Проблема и Решение}
	\begin{columns}
		\column{0.5\textwidth}
		\textbf{Что было не так?}
		\begin{itemize}
			\item Ручная обработка операций выдачи и возврата книг
			\item Отсутствие централизованного каталога
			\item Сложность отслеживания состояния библиотечных материалов
			\item Неэффективное управление библиотечными процессами
		\end{itemize}
		
		\column{0.5\textwidth}
		\textbf{Что мы предлагаем?}
		\begin{itemize}
			\item Автоматизация процессов выдачи и возврата
			\item Централизованный каталог библиотечных материалов
			\item Онлайн-бронирование и продление сроков выдачи
			\item Удобный доступ читателей к библиотечным ресурсам
		\end{itemize}
	\end{columns}
\end{frame}

% Слайд 3: Для кого и зачем?
\begin{frame}{Для кого и зачем?}
	\textbf{Ключевые стейкхолдеры:}
	\begin{itemize}
		\item \textbf{Читатели} --- пользователи библиотеки, которые могут искать книги, бронировать и продлевать сроки выдачи
		\item \textbf{Библиотекари} --- сотрудники библиотеки, которые управляют каталогом и обрабатывают операции выдачи/возврата
		\item \textbf{Директор библиотеки} --- заказчик системы, определяет стратегические цели
	\end{itemize}
	
	\vspace{0.5cm}
	
	\textbf{Главное бизнес-требование (SMART):}
	\begin{center}
		\large
		\textbf{Увеличить среднее количество бронирований на пользователя на \textcolor{blue}{15\%} через персональные рекомендации и напоминания}
	\end{center}
\end{frame}

% Слайд 4: Как это работает? (BPMN)
\begin{frame}{Как это работает? (BPMN)}
	\begin{center}
		\includegraphics[width=0.65\textwidth]{Lab2_sa_sd/Lab2_sa_sd.png}
	\end{center}
	
	\vspace{0.2cm}
	
	\textbf{Логика процесса выдачи книги:}
	\begin{enumerate}
		\item Библиотекарь идентифицирует пользователя в системе
		\item Система проверяет доступность материала и ограничения пользователя
		\item При отсутствии ограничений система регистрирует выдачу
		\item Система формирует уведомление пользователю о выданном материале
	\end{enumerate}
\end{frame}

% Слайд 5: Из чего это состоит? (Архитектура)
\begin{frame}{Из чего это состоит? (Архитектура)}
	\begin{center}
		\includegraphics[width=0.7\textwidth]{Lab4_sa_sd/C2-diagram.png}
	\end{center}
\end{frame}

% Слайд 6: Выбор технологий
\begin{frame}{Выбор технологий}
	\footnotesize
	Диаграмма контейнеров показывает высокоуровневую архитектуру системы, включая основные приложения, базы данных и внешние системы.
	
	\vspace{0.1cm}
	
	\textbf{Технологический стек:}
	\begin{columns}
		\column{0.5\textwidth}
		\textbf{Frontend:} React + Redux + TypeScript
		\begin{itemize}
			\setlength{\itemsep}{0pt}
			\setlength{\parskip}{0pt}
			\item Компонентный подход, типобезопасность
			\item Большое сообщество, богатая экосистема
		\end{itemize}
		
		\vspace{0.05cm}
		
		\textbf{Backend:} Java + Spring REST
		\begin{itemize}
			\setlength{\itemsep}{0pt}
			\setlength{\parskip}{0pt}
			\item Зрелый фреймворк, надёжность
			\item Отличная поддержка REST API
		\end{itemize}
		
		\vspace{0.05cm}
		
		\textbf{База данных:} PostgreSQL
		\begin{itemize}
			\setlength{\itemsep}{0pt}
			\setlength{\parskip}{0pt}
			\item ACID-транзакции, надёжность
			\item Открытый исходный код
		\end{itemize}
		
		\column{0.5\textwidth}
		\textbf{Инфраструктура:}
		\begin{itemize}
			\setlength{\itemsep}{0pt}
			\setlength{\parskip}{0pt}
			\item \textbf{Redis} --- быстрый кэш запросов
			\item \textbf{RabbitMQ} --- асинхронная обработка уведомлений
			\item \textbf{ELK stack} --- централизованное логирование
		\end{itemize}
		
		\vspace{0.05cm}
		
		\textbf{Внешние сервисы:}
		\begin{itemize}
			\setlength{\itemsep}{0pt}
			\setlength{\parskip}{0pt}
			\item Платёжный сервис (HTTPS/REST)
			\item Почтовый/SMS-сервис (SMTP/HTTPS)
		\end{itemize}
	\end{columns}
	
	\vspace{0.1cm}
	
	\begin{center}
		\tiny
		Выбранный стек обеспечивает масштабируемость, надёжность и безопасность системы.
	\end{center}
\end{frame}

% Слайд 7: Что умеет система? (Use Case Diagram)
\begin{frame}{Что умеет система? (Use Case Diagram)}
	\begin{center}
		\includegraphics[width=0.85\textwidth]{Lab3_sa_sd/Lab3_sa_sd.png}
	\end{center}
\end{frame}

% Слайд 8: Функциональные и нефункциональные требования
\begin{frame}{Функциональные и нефункциональные требования}
	\begin{columns}
		\column{0.5\textwidth}
		\textbf{Ключевые функции:}
		\begin{itemize}
			\item \textbf{Авторизация} --- регистрация и вход пользователей
			\item \textbf{Поиск материалов} --- поиск книг в каталоге
			\item \textbf{Оформление брони} --- резервирование материалов
			\item \textbf{Оформление выдачи} --- выдача материалов читателям
			\item \textbf{Возврат материала} --- обработка возврата
			\item \textbf{Погашение штрафа} --- оплата штрафов онлайн
		\end{itemize}
		
		\column{0.5\textwidth}
		\textbf{Нефункциональные требования:}
		\begin{itemize}
			\item Поисковые запросы обрабатываются за время \textbf{не более 1.5 секунд} (95\% запросов)
			\item Доступность системы не менее \textbf{98\%} в течение месяца
			\item Система поддерживает до \textbf{200 одновременных пользователей}
			\item Все данные защищены протоколом \textbf{TLS 1.2+}
		\end{itemize}
	\end{columns}
\end{frame}

% Слайд 9: Итоги
\begin{frame}{Итоги}
	\begin{center}
		\large
		\textbf{Мы спроектировали систему управления библиотекой, которая:}
	\end{center}
	
	\vspace{0.3cm}
	
	\begin{itemize}
		\item Решает проблему неэффективного управления библиотечными процессами
		\item Автоматизирует операции выдачи и возврата книг
		\item Предоставляет удобный доступ к библиотечным ресурсам
		\item Обеспечивает централизованный каталог материалов
	\end{itemize}
	
	\vspace{0.3cm}
	
	\begin{center}
		\large
		\textbf{Все требования и архитектура документированы}
		
		\vspace{0.5cm}
		
		\LARGE
		\textbf{Спасибо за внимание!}
		
		\vspace{0.3cm}
		
		\large
		\textbf{Вопросы?}
	\end{center}
\end{frame}

\end{document}

