\documentclass[12pt]{report}

\usepackage{cmap}
\usepackage[T1,T2A]{fontenc}
\usepackage[utf8]{inputenc}
\usepackage[english, russian]{babel}
\usepackage{amssymb}
\usepackage{amsmath}
\usepackage{amsthm}
\usepackage{dsfont}
\usepackage{bm}
\usepackage{diagbox}
\usepackage{array}
\usepackage{placeins}
\usepackage[left=20mm,right=10mm,top=20mm,bottom=20mm,bindingoffset=2mm]{geometry}
\usepackage{indentfirst}
\usepackage[utf8]{inputenc}
\usepackage{float}
\usepackage[hidelinks]{hyperref}
\usepackage{graphicx}
\usepackage{xcolor}
\usepackage{listings}
\usepackage{minted}

\DeclareMathOperator{\N}{\mathbb{N}}
\DeclareMathOperator{\R}{\mathbb{R}}
\DeclareMathOperator{\Z}{\mathbb{Z}}
\DeclareMathOperator{\CC}{\mathbb{C}}
\DeclareMathOperator{\PP}{\mathrm{P}}
\DeclareMathOperator{\Expec}{\mathrm{E}}
\DeclareMathOperator{\Var}{\mathrm{Var}}
\DeclareMathOperator{\Cov}{\mathrm{Cov}}
\DeclareMathOperator{\asConv}{\xrightarrow{a.s.}}
\DeclareMathOperator{\LpConv}{\xrightarrow{Lp}}
\DeclareMathOperator{\pConv}{\xrightarrow{p}}
\DeclareMathOperator{\dConv}{\xrightarrow{d}}

\hypersetup{
	colorlinks=true,
	linkcolor=blue,
	citecolor=blue,
	urlcolor=blue
}

\lstset{language=Python, extendedchars=\true}

\lstdefinestyle{pythonstyle}{
	language=Python,
	backgroundcolor=\color{lightgray},
	commentstyle=\color{green},
	keywordstyle=\color{blue},
	stringstyle=\color{red},
	basicstyle=\ttfamily,
	frame=single,
	breaklines=true
}

\addto\captionsrussian{\renewcommand{\refname}{Список использованных источников}}

\begin{document}
	
	\begin{titlepage}
		\begin{center}
			\large{Федеральное государственное автономное образовательное учреждение высшего образования <<Национальный исследовательский университет ИТМО>>}
		\end{center}
		
		\vspace{15em}
		
		\begin{center}
			\huge{\textbf{Проектная работа}} \\
			\large{По дисциплине <<Системный анализ и системное проектирование>>} \\
			\large{Итерация №4}
		\end{center}
		
		\vspace{2em}
		
		\begin{flushright}
			\textit{\large{Выполнили:}} \\
			\large{Студент группы P3306} \\
			\large{Михайлов Дмитрий} \\
			\large{Андреевич} \\
			
			\large{Студент группы P3312} \\
			\large{Малышев Никита} \\
			\large{Александрович} \\
			\textit{\large{Преподаватель:}} \\
			\large{Маркина Татьяна} \\
			\large{Анатольевна}
		\end{flushright}
		
		\vspace{2cm}
		
		\begin{figure}[h]
			\centering
			\includegraphics[width=0.5\linewidth]{image.png}
		\end{figure}
		
		\begin{center}
			Санкт-Петербург \\
			2025 год
		\end{center}
	\end{titlepage}
	
	\tableofcontents
	\newpage
	
	\addcontentsline{toc}{section}{Диаграмма контекста}
	\section*{Диаграмма контекста}
	
	\begin{figure}[h]
		\centering
		\includegraphics[width=0.6\textwidth]{C1-diagram.png}
		\caption{Диаграмма Контекста (Context Diagram - C1).}
		\label{fig:Context Diagram}
	\end{figure}
	\newpage
	
	\addcontentsline{toc}{section}{Диаграмма контейнеров}
	\section*{Диаграмма контейнеров}
	
	\begin{figure}[h]
		\centering
		\includegraphics[width=1\textwidth]{C2-diagram.png}
		\caption{Диаграмма Контейнеров (Container Diagram - C2).}
		\label{fig:Container Diagram}
	\end{figure}
	\newpage
	
	\addcontentsline{toc}{section}{Диаграмма компонентов}
	\section*{Диаграмма компонентов}
	
	\begin{figure}[h]
		\centering
		\includegraphics[width=1\textwidth]{C3-diagram.png}
		\caption{Диаграмма Компонентов (Component Diagram - C3) для одного ключевого контейнера.}
		\label{fig:Component Diagram}
	\end{figure}
	\newpage
	
	\addcontentsline{toc}{section}{Диаграмма классов}
	\section*{Диаграмма классов}
	
	\begin{figure}[h]
		\centering
		\includegraphics[width=1\textwidth]{C4-diagram.png}
		\caption{Диаграмма классов (Class Diagram - C4).}
		\label{fig:Class Diagram}
	\end{figure}
	\newpage
	
	\addcontentsline{toc}{section}{Диаграмма последовательностей}
	\section*{Диаграмма последовательностей}
	
	\begin{figure}[h]
		\centering
		\includegraphics[width=0.7\textwidth]{Sequence Diagram.png}
		\caption{Диаграмма последовательностей (Sequence Diagram).}
		\label{fig:Sequence Diagram}
	\end{figure}
	
\end{document}
