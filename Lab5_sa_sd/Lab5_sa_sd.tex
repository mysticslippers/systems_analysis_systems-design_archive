\documentclass[12pt]{report}

\usepackage{cmap}
\usepackage[T1,T2A]{fontenc}
\usepackage[utf8]{inputenc}
\usepackage[english, russian]{babel}
\usepackage{amssymb}
\usepackage{amsmath}
\usepackage{amsthm}
\usepackage{dsfont}
\usepackage{bm}
\usepackage{diagbox}
\usepackage{array}
\usepackage{placeins}
\usepackage[left=20mm,right=10mm,top=20mm,bottom=20mm,bindingoffset=2mm]{geometry}
\usepackage{indentfirst}
\usepackage[utf8]{inputenc}
\usepackage{float}
\usepackage[hidelinks]{hyperref}
\usepackage{graphicx}
\usepackage{xcolor}
\usepackage{listings}
\usepackage{minted}

\DeclareMathOperator{\N}{\mathbb{N}}
\DeclareMathOperator{\R}{\mathbb{R}}
\DeclareMathOperator{\Z}{\mathbb{Z}}
\DeclareMathOperator{\CC}{\mathbb{C}}
\DeclareMathOperator{\PP}{\mathrm{P}}
\DeclareMathOperator{\Expec}{\mathrm{E}}
\DeclareMathOperator{\Var}{\mathrm{Var}}
\DeclareMathOperator{\Cov}{\mathrm{Cov}}
\DeclareMathOperator{\asConv}{\xrightarrow{a.s.}}
\DeclareMathOperator{\LpConv}{\xrightarrow{Lp}}
\DeclareMathOperator{\pConv}{\xrightarrow{p}}
\DeclareMathOperator{\dConv}{\xrightarrow{d}}

\hypersetup{
	colorlinks=true,
	linkcolor=blue,
	citecolor=blue,
	urlcolor=blue
}

\lstset{language=Python, extendedchars=\true}

\lstdefinestyle{pythonstyle}{
	language=Python,
	backgroundcolor=\color{lightgray},
	commentstyle=\color{green},
	keywordstyle=\color{blue},
	stringstyle=\color{red},
	basicstyle=\ttfamily,
	frame=single,
	breaklines=true
}

\addto\captionsrussian{\renewcommand{\refname}{Список использованных источников}}

\begin{document}
	
	\begin{titlepage}
		\begin{center}
			\large{Федеральное государственное автономное образовательное учреждение высшего образования <<Национальный исследовательский университет ИТМО>>}
		\end{center}
		
		\vspace{15em}
		
		\begin{center}
			\huge{\textbf{Проектная работа}} \\
			\large{По дисциплине <<Системный анализ и системное проектирование>>} \\
			\large{Итерация №5}
		\end{center}
		
		\vspace{2em}
		
		\begin{flushright}
			\textit{\large{Выполнили:}} \\
			\large{Студент группы P3306} \\
			\large{Михайлов Дмитрий} \\
			\large{Андреевич} \\
			
			\large{Студент группы P3312} \\
			\large{Малышев Никита} \\
			\large{Александрович} \\
			\textit{\large{Преподаватель:}} \\
			\large{Маркина Татьяна} \\
			\large{Анатольевна}
		\end{flushright}
		
		\vspace{2cm}
		
		\begin{figure}[h]
			\centering
			\includegraphics[width=0.5\linewidth]{image.png}
		\end{figure}
		
		\begin{center}
			Санкт-Петербург \\
			2025 год
		\end{center}
	\end{titlepage}
	
	\tableofcontents
	\newpage
	
	\addcontentsline{toc}{section}{Нефункциональные требования (NFR)}
	\section*{Нефункциональные требования (NFR)}
	
	\subsection{Производительность}
	\begin{enumerate}
		\item 95\% поисковых запросов к каталогу должны выполняться менее чем за 1 секунду при нагрузке до 500 одновременных пользователей.
		\item Среднее время обработки выдачи или возврата книги не должно превышать 2 секунд.
		\item Формирование отчета библиотекаря о выданных книгах не должно занимать более 5 секунд для библиотеки с 100\,000 записей.
		\item Продление срока книги пользователем должно выполняться менее чем за 2 секунды.
		\item Обновление информации о книге библиотекарем (например, изменение статуса или добавление новой книги) должно занимать не более 3 секунд.
		\item Отображение результатов поиска по фильтрам (автор, жанр, год издания) должно занимать менее 1.5 секунд.
	\end{enumerate}
	
	\subsection{Надежность и доступность}
	\begin{enumerate}
		\item Система должна обеспечивать uptime не менее 99.5\% в течение месяца.
		\item Время восстановления работы после критического сбоя не должно превышать 1 час.
		\item Автоматическое ежедневное резервное копирование должно выполняться в ночное время без прерывания работы сервиса.
		\item Данные должны восстанавливаться из последней резервной копии за не более 30 минут.
		\item Система должна корректно обрабатывать до 3 одновременных отказов отдельных компонентов без потери данных.
		\item Все транзакции по выдаче и возврату книг должны быть атомарными и сохраняться даже при сбое сервера.
	\end{enumerate}
	
	\subsection{Безопасность}
	\begin{enumerate}
		\item Все пароли пользователей должны храниться в хэшированном виде с использованием алгоритма bcrypt с солью.
		\item Доступ к управлению пользователями и каталогами материалов должен быть ограничен ролями (библиотекарь, администратор).
		\item Все данные при передаче между клиентом и сервером должны быть зашифрованы с использованием протокола TLS 1.2 или выше.
		\item Все действия библиотекаря и администратора должны логироваться с отметкой времени и идентификатором пользователя.
		\item Система должна блокировать учетную запись пользователя после 5 неудачных попыток авторизации в течение 15 минут.
		\item Все критические операции (удаление записи, начисление штрафа) должны требовать подтверждения пользователя и фиксироваться в журнале аудита.
	\end{enumerate}
	
	\subsection{Удобство использования (Usability)}
	\begin{enumerate}
		\item Новый пользователь должен зарегистрироваться и авторизоваться за не более чем 2 минуты.
		\item Пользователь должен оформить первую выдачу или бронирование не более чем за 3 минуты.
		\item Продление книги должно быть доступно в 2 клика для зарегистрированного пользователя.
		\item Библиотекарь должен иметь возможность добавить новую книгу или пользователя за не более 1 минуты.
		\item Интерфейс поиска должен отображать результаты с релевантностью не менее 90\% по введенному запросу.
		\item Пользователь должен получить уведомление о просроченной книге не позднее 1 часа после наступления срока возврата.
	\end{enumerate}
	
	\subsection{Масштабируемость}
	\begin{enumerate}
		\item Система должна корректно работать при одновременном подключении до 1000 пользователей без деградации производительности.
		\item Подключение новой библиотеки должно занимать не более 24 часов без остановки сервиса.
		\item Система должна поддерживать до 1 миллиона записей книг без замедления основных операций более чем на 10\%.
		\item Количество одновременных операций выдачи/возврата не должно снижать скорость обработки поиска более чем на 15\%.
		\item Масштабирование серверной инфраструктуры должно быть автоматическим при росте нагрузки выше 80\% CPU/памяти.
		\item Система должна обеспечивать добавление новых функций (например, новые типы материалов) без остановки работы существующих модулей.
	\end{enumerate}
	
	\subsection{Поддерживаемость}
	\begin{enumerate}
		\item Документация для пользователей и библиотекарей должна обновляться при каждом изменении функционала и быть доступна онлайн.
		\item Время внедрения нового функционала не должно превышать 2 недель с момента утверждения требований.
		\item Система должна предоставлять возможность мониторинга ошибок и сбоев в реальном времени через админ-панель.
		\item Любое изменение данных библиотекаря должно логироваться для возможности отката.
		\item Модификация существующих модулей не должна нарушать работу системы более чем на 1 час в тестовом окружении.
		\item Все изменения в системе должны проходить автоматическое тестирование с покрытием не менее 80\% критических функций.
	\end{enumerate}
	
\end{document}
