\documentclass[12pt]{report}

\usepackage{cmap}
\usepackage[T1,T2A]{fontenc}
\usepackage[utf8]{inputenc}
\usepackage[english, russian]{babel}
\usepackage{amssymb}
\usepackage{amsmath}
\usepackage{amsthm}
\usepackage{dsfont}
\usepackage{bm}
\usepackage{diagbox}
\usepackage{array}
\usepackage{placeins}
\usepackage[left=20mm,right=10mm,top=20mm,bottom=20mm,bindingoffset=2mm]{geometry}
\usepackage{indentfirst}
\usepackage[utf8]{inputenc}
\usepackage{float}
\usepackage[hidelinks]{hyperref}
\usepackage{graphicx}
\usepackage{xcolor}
\usepackage{listings}
\usepackage{minted}

\DeclareMathOperator{\N}{\mathbb{N}}
\DeclareMathOperator{\R}{\mathbb{R}}
\DeclareMathOperator{\Z}{\mathbb{Z}}
\DeclareMathOperator{\CC}{\mathbb{C}}
\DeclareMathOperator{\PP}{\mathrm{P}}
\DeclareMathOperator{\Expec}{\mathrm{E}}
\DeclareMathOperator{\Var}{\mathrm{Var}}
\DeclareMathOperator{\Cov}{\mathrm{Cov}}
\DeclareMathOperator{\asConv}{\xrightarrow{a.s.}}
\DeclareMathOperator{\LpConv}{\xrightarrow{Lp}}
\DeclareMathOperator{\pConv}{\xrightarrow{p}}
\DeclareMathOperator{\dConv}{\xrightarrow{d}}

\hypersetup{
	colorlinks=true,
	linkcolor=blue,
	citecolor=blue,
	urlcolor=blue
}

\lstset{language=Python, extendedchars=\true}

\lstdefinestyle{pythonstyle}{
	language=Python,
	backgroundcolor=\color{lightgray},
	commentstyle=\color{green},
	keywordstyle=\color{blue},
	stringstyle=\color{red},
	basicstyle=\ttfamily,
	frame=single,
	breaklines=true
}

\addto\captionsrussian{\renewcommand{\refname}{Список использованных источников}}

\begin{document}
	
	\begin{titlepage}
		\begin{center}
			\large{Федеральное государственное автономное образовательное учреждение высшего образования <<Национальный исследовательский университет ИТМО>>}
		\end{center}
		
		\vspace{15em}
		
		\begin{center}
			\huge{\textbf{Проектная работа}} \\
			\large{По дисциплине <<Системный анализ и системное проектирование>>} \\
			\large{Итерация №5}
		\end{center}
		
		\vspace{2em}
		
		\begin{flushright}
			\textit{\large{Выполнили:}} \\
			\large{Студент группы P3306} \\
			\large{Михайлов Дмитрий} \\
			\large{Андреевич} \\
			
			\large{Студент группы P3312} \\
			\large{Малышев Никита} \\
			\large{Александрович} \\
			\textit{\large{Преподаватель:}} \\
			\large{Маркина Татьяна} \\
			\large{Анатольевна}
		\end{flushright}
		
		\vspace{2cm}
		
		\begin{figure}[h]
			\centering
			\includegraphics[width=0.5\linewidth]{image.png}
		\end{figure}
		
		\begin{center}
			Санкт-Петербург \\
			2025 год
		\end{center}
	\end{titlepage}
	
	\tableofcontents
	\newpage
	
	\addcontentsline{toc}{section}{Нефункциональные требования (NFR)}
	\section*{Нефункциональные требования (NFR)}
	
	\subsection{Производительность}
	\begin{enumerate}
		\item Система каталога должна обрабатывать не менее 95\% поисковых запросов по каталогу при нагрузке до 100 одновременных пользователей за время не более 1.5 секунд в 95\% измерений за календарные сутки.
		\item Система обработки операций выдачи должна выполнять регистрацию выдачи или возврата одной книги при нагрузке до 100 одновременных библиотекарей за время не более 2 секунд в 95\% измерений.
		\item Подсистема отчётности должна формировать отчёт о выданных книгах по одной библиотеке объёмом до 100\,000 записей за время не более 5 секунд при нормальной нагрузке.
		\item Подсистема продления должна регистрировать продление срока одной выдачи пользователем при нагрузке до 200 запросов в минуту за время не более 2 секунд.
		\item Подсистема управления каталогом должна сохранять изменения карточки одной книги (редактирование полей или добавление новой книги) при нагрузке до 50 одновременных библиотекарей за время не более 3 секунд.
		\item Подсистема поиска с фильтрами должна возвращать пользователю результаты поиска с применёнными фильтрами (автор, жанр, год издания) при объёме каталога до 100\,000 записей за время не более 2 секунд.
	\end{enumerate}
	
	\subsection{Надёжность и доступность}
	\begin{enumerate}
		\item Система в целом должна обеспечивать доступность (uptime) не менее 98\% в течение календарного месяца при эксплуатации в штатном режиме.
		\item Подсистема восстановления должна восстанавливать работоспособность системы после критического сбоя за время не более 2 часов с момента фиксации сбоя в системе мониторинга.
		\item Подсистема резервного копирования должна выполнять ежедневное полное резервное копирование базы данных объёмом до 10~ГБ в промежутке с 02{:}00 до 04{:}00 по серверному времени без остановки пользовательских операций более чем на 2 минуты.
		\item Подсистема восстановления данных должна восстанавливать базу данных из последней успешной резервной копии объёмом до 10~ГБ за время не более 1 часа.
	\end{enumerate}
	
	\subsection{Безопасность}
	\begin{enumerate}
		\item Подсистема аутентификации должна хранить пароли пользователей только в виде хэш-сумм, рассчитанных алгоритмом bcrypt с индивидуальной солью длиной не менее 16 байт во всех пользовательских записях.
		\item Подсистема авторизации должна разрешать доступ к операциям управления пользователями и каталогом материалов только учетным записям с ролями ``Библиотекарь'' или ``Администратор'' при проверке прав в каждой защищаемой операции.
		\item Транспортный уровень системы должен обеспечивать шифрование всех запросов и ответов между клиентом и сервером с использованием протокола TLS версии не ниже 1.2 во 100\% внешних соединений.
		\item Подсистема аудита должна записывать в журнал аудита все действия библиотекарей и администраторов (создание, изменение, удаление сущностей) с указанием идентификатора пользователя, времени и типа операции для 100\% таких действий.
		\item Подсистема защиты от перебора паролей должна блокировать учётную запись пользователя на 15 минут после 5 подряд неуспешных попыток входа, совершённых в интервале 15 минут.
		\item Подсистема подтверждения критических операций должна требовать явного подтверждения пользователя и записывать в журнал аудита каждую операцию удаления материалов, начисления штрафов и изменения статуса библиотек в 100\% случаев.
	\end{enumerate}
	
	\subsection{Удобство использования (Usability)}
	\begin{enumerate}
		\item Пользовательский веб-интерфейс должен позволять новому пользователю пройти регистрацию и авторизацию при наличии стабильного соединения за время не более 3 минут в 90\% сценариев юзабилити-тестирования.
		\item Пользовательский интерфейс должен позволять читателю оформить первую бронь или выдачу книги при знании названия или автора за время не более 4 минут в 90\% тестовых сценариев.
		\item Интерфейс личного кабинета должен позволять зарегистрированному читателю продлить срок выдачи книги не более чем в два клика мышью (или тапа) без перехода на дополнительные страницы.
		\item Интерфейс библиотекаря должен позволять добавить карточку новой книги с заполнением обязательных полей за время не более 1 минуты при наличии готовых данных в 95\% тестовых сценариев.
		\item Подсистема поиска должна обеспечивать, чтобы не менее 90\% книг, считающихся релевантными экспертами по введённому ключевому слову, отображались на первой странице результатов поиска.
		\item Подсистема уведомлений должна доставлять электронное уведомление о наступлении просрочки возврата книги пользователю в течение не более 1 часа после наступления даты возврата при доступности внешнего сервиса уведомлений.
	\end{enumerate}
	
	\subsection{Масштабируемость}
	\begin{enumerate}
		\item Система приложений должна поддерживать одновременную работу не менее 200 активных пользователей при среднем количестве 3 запросов в минуту от одного пользователя без увеличения среднего времени ответа основных операций более чем на 20\% относительно базового уровня.
		\item Подсистема подключения библиотек должна позволять добавить одну новую библиотеку с заполнением конфигурационных данных и активацией в системе за время не более 24 часов без полной остановки сервиса.
		\item Подсистема хранения данных должна поддерживать объём каталога до 100\,000 записей материалов без роста среднего времени выполнения поискового запроса более чем до 2 секунд.
		\item Система обработки операций выдачи и возврата должна выдерживать до 50 одновременных операций выдачи/возврата в секунду без снижения средней скорости ответа поиска по каталогу более чем на 15\%.
		\item Инфраструктурный слой системы должен автоматически масштабироваться горизонтально (добавление экземпляров приложений) при среднем использовании CPU выше 80\% в течение 5 минут, снижая использование до уровня ниже 70\% в течение последующих 10 минут.
		\item Архитектура приложения должна обеспечивать возможность добавления новых функциональных модулей (например, поддержки новых типов материалов) путём развертывания отдельных сервисов без остановки существующих модулей более чем на 5 минут для одной библиотеки.
	\end{enumerate}
	
	\subsection{Поддерживаемость}
	\begin{enumerate}
		\item Подсистема документации должна публиковать обновлённую пользовательскую и административную документацию в онлайн-доступе не позднее 3 рабочих дней после утверждения изменения функциональных требований.
		\item Процесс внедрения новой функциональности в промышленное окружение должен занимать не более 2 недель с момента утверждения спецификации в 90\% релизов.
		\item Подсистема мониторинга должна отображать информацию о критических ошибках и сбоях всех сервисов в админ-панели с задержкой не более 1 минуты от момента возникновения события.
		\item Подсистема аудита должна фиксировать любые изменения данных библиотекарем (создание, изменение, удаление материалов и читателей) в журнале аудита в 100\% случаев, обеспечивая возможность отката на уровне базы данных.
		\item Процесс модификации существующего модуля и его развёртывания в тестовом окружении не должен приводить к полной недоступности тестовой среды более чем на 1 час за один релиз.
		\item Система автоматического тестирования должна обеспечивать покрытие автотестами не менее 80\% критических пользовательских сценариев (поиск, бронирование, выдача, возврат, оплата штрафов), измеряемое метрикой покрытия по строкам кода и сценариям.
	\end{enumerate}
	
	
\end{document}
